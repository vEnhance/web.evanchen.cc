\documentclass[11pt]{scrreprt}
\usepackage[sexy,chinese]{evan}

\begin{document}
\title{2014 臺灣 IMO}
\subtitle{A Brief Reflection}
\date{1 August 2014}
\maketitle

\begin{abstract}
  This report describes my experiences as a contestant for the Taiwanese IMO 2014 team.
  I draw several comparisons between the Taiwanese selection camps and the USA training camp (MOP).
  It is mainly a personal diary for my reflection, but hopefully will also be insightful for others.
\end{abstract}

\tableofcontents

\chapter{Overview}
\section{Description of Selection Process}
\subsection{Camps}
The top 25-30 finishers of the Asian-Pacific Math Olympiad (qualification occurring through previous tests which I am not familiar with) are invited to the first of three training camps (段選訓營), held around April each year.  The camps generally begin on a Friday and end on a Tuesday.

The camps gradually decrease in size as contestants are eliminated. In 2014, the camp dates were as follows:
\begin{enumerate}
  \ii March 28 -- April 1; 28 contestants.
  \ii April 11 -- April 15; 15 contestants.
  \ii April 25 -- April 30; 10 contestants.
\end{enumerate}

I attended the first and third camps only, and was allowed to take the tests for the second camp remotely.

After the team is selected, five additional training camps are held throughout May and June. In 2014, the dates are
\begin{enumerate}
  \ii May 9 -- May 13
  \ii May 23 -- May 27
  \ii June 6 -- June 10
  \ii June 20 -- June 23
  \ii June 27 -- June 30
\end{enumerate}
Finally, the 55th IMO 2014 will last from July 5 through July 14th. This year's IMO will be held in Cape Town, South Africa.

\subsection{Test Format and Selection Index}
Each camp consists of three quizzes (獨立研究) and a two-day Team Selection Test (模擬競賽).
A quiz lasts 110 minutes and contains two problems, ostensibly somewhat easier. The TST follows the format of the IMO.

Needless to say, every problem is scored out of 7 points.

Advancement to the second stage from the first consists of a weighted average of the quizzes and tests from the first camp, plus an APMO index given by
\[ \max \left\{ \text{APMO 2014}, 0.8 \times \text{APMO 2013} \right\}. \]

However, the first camp scores are not used in the IMO selection.

The IMO index is based on the following:
\begin{itemize}
  \ii 11\% Quizzes from Camp 2
  \ii 35\% TST from Camp 2
  \ii 1\% Miscellaneous (Camp 2)
  \ii 11\% Quizzes from Camp 3
  \ii 35\% TST from Camp 3
  \ii 7\% Oral Exam (Camp 3)
\end{itemize}
The ``miscellaneous'' is not particularly meaningful here, and my impression is that it mostly is to help persuade people to go to bed at reasonable hours.
The oral exam seems to differ from year to year, and is given on the last day of the third camp.


\section{Synopsis of Camp}
The schedule is really very packed; I was tired at the end of every day.

The activities fall roughly into three categories.
\begin{itemize}
  \ii Quiz or Test
  \ii Subject Lecture
  \ii Discussion / Practice Session
\end{itemize}

I've already described the format of the quizzes and tests above. One interesting thing to note was that English translations of the problems were also provided, probably for my sake.

By ``discussion'' I mean the entire group is placed in a classroom for an hour or two and works on problems. Sometimes it is entirely anarchy; other times there is an instructor or previous contestant who prepares the problems for everyone to work on. As I mention below, this time was actually very well-used by the majority of the students, and I had a really fun time teaching projective transformations to a large group of other students around one of the white boards.

There were a lot of cryptic entries on the schedule that I didn't understand, like 習作, 選訓專題, 個別評量, and so on. These seem to roughly correspond to ``subject lecture continues'' or ``free discussion'', which is probably not necessarily a bad thing.

I explain a lot more of this in detail in the diary of events.

\section{Main Impressions and Comments}
The thing which impressed me the most was the diligence of the Taiwanese students. During the day, they had a much stronger tendency to be practicing math with each other (as opposed to the US kids, who like to play cards during their free time) or chatting about problems (and they definitely enjoyed it). I think peers are among the most valuable resource the students have, because it is really important to see how other students solve problems, especially ones that you do not. The Taiwan students used this opportunity very well.

The team selection quizzes were a new idea to me, but my feeling is that they were much more time-efficient in terms of training than the full mock IMO, in that they do a nice job of simulating the time pressure of the olympiad without actually taking too much time from the camp.

I also think that the score-prediction after each exam was good, both for helping students learn what a full $7$ means (and in particular learning what kind of things can confuse graders), as well as to make sure there are no errors in grading\footnote{The US system is a lot more closed for the actual TST -- often we are never told our scores.}.
It's also nice to be able to physically talk with the grader, as it becomes more naturally obvious what parts of the solution were confusing.

My main criticism of the lectures is that in a lot of them I didn't feel like I had a chance to work on any problems myself, either because the solutions were presented shortly after the problems, or because there were no problems at all. I remember that in the first geometry lecture, we had a handout with the problems that were being discussed, and most of the students ignored the teacher (who was presenting solutions) and just worked on the problems together. After the third camp, I felt a lot less strongly about this when I realized that there was plenty of problem-solving being done in the practice sessions, but nonetheless I think the point still stands that there should be some space between seeing a problem in the lecture and seeing the solution. (And either way, there should be problems.)

To elaborate on the above, I don't think there's anything wrong with the instructor discussing solutions, per se. In fact I think that the leaders / deputy leaders / observers often have a lot of insightful things to say about the solutions that the students can't obviously see (for example, it is really helpful to know what the thought process of the IMO committee was when they selected a certain problem). The US doesn't do much of this at all and I would like to see more of it. I think the real issue comes if you don't give students any time to think about the problem before discussing the solution, because then it's really hard to see where the motivation for the solution comes from.

Fortunately this is really easy to fix. The laziest way to do so is to simply send out the problems beforehand and say ``these are the problems we'll be discussing in this week's lecture; please try them briefly before coming to the lecture''. You'd have to count on the students to be diligent enough to actually do the problems beforehand, so this wouldn't work well in the United States, but it might fly in Taiwan. The other way of course is to just let the students work on the problems for the first 15-25 minutes of lecture.

The other criticism I have was the elimination process between the camps, which I think is too rapid. One part is that the students seemed constantly worried at the first camp about whether they would advance to the next level, which I think is not really great for their learning. The other issue is that it hurts the future teams. The USA has a strong pool in part because they train about 60 students for three weeks, even though most of them never eventually make the USA IMO team (like me). Hence a student who eventually does attend the IMO has usually been to training for three or more years. In contrast it seems easy for a Taiwan contestant to keep being eliminated in early rounds until the last few years of high school, limiting the amount of training they get.

Socially, the TST camp was a great experience. Like in the US, I considered everyone I met there to be a good friend, and thank God for Facebook because I'm still keeping in touch with them.
The top students in the US are really good about not worrying too much about out-competing one another so much as everyone just trying to do their absolute best and have a good time.
I'm happy to report that the same is true in Taiwan, and I hope I continue to bump into my classmates as the years wear on.

Arbitrary side remark, the Taiwan students (this year) are quite fond of geometry overall.


\chapter{Camp Diary}
I attended the first and third camps.

\section{First Camp}
\subsection{March 28}
Everyone seems to know each other on arrival. I mostly lurk, watching quietly, and not really understanding all the Chinese around me. I get my name card (14MC02) and room key, set down my belongings, and mostly look forward to the start of the math.

In the geometry lecture we're presented with a bunch of problems; I don't recall the instructor arriving for a while, so I had a while to crack at some of them before the lecture actually began. Some of the problems were easy; others were rather hard, namely the first two. The lecture consisted of the teacher talking about the problems with the aid of a PowerPoint presentation (in Chinese, which did not help me). There were some comments he made that I remember being useful, although I don't remember what now. Nevertheless, I think the majority of people ignored the teacher and just worked together on the handouts. I tuned out after the first two problems were presented as I had already solved all the remaining problems and only briefly glanced at the solutions.

Ironically, we are presented with good American-style hamburgers at a restaurant for dinner. When I sit alone at a table I'm invited to join the others by some of the other students, but for the most part I can't really communicate with them and just watch. It does seem that they know each other pretty well.

Afterwards we had the first quiz; thereafter I retire to bed, much earlier than my roommate, and I do not wake when he enters the room some hours later.

\subsection{March 29}
Nerdily enough, we eat breakfast in the classroom. I personally think this is a great idea as I dislike having to move from dining halls to classrooms, and also enjoy having a desk to work at during these times. I spend the majority of this morning texting Jessica.

When the algebra lecture begins, we're assigned into groups (ours being named 空白 because we failed to produce a name), and immediately presented with the classical $15$ problem involving parity of permutations. After some time passes while the instructor is setting up, I explain the solution to the other people in my group, with some rather broken Chinese and Wikipedia on my phone.

When the lecture begins I expect to learn about group theory, but instead we compete in our groups to solve the 15 puzzle at the front. This probably didn't help our olympiad scores much, but it was a lot of fun and I don't mind considering how packed the schedule is. I am a little annoyed that this ends up taking about half an hour.
During the next half hour, we are presented with the Rearrangement Inequality, and asked to prove AM-GM, Cauchy, and Chebyshev using it. We are to write up our solutions and pass it to another group for ``grading''.
No one in my group really knows how to do any of them other than Chebyshev, but I write up a solution to AM-GM and Cauchy along with the team's Chebyshev solution, and the other group gives us full marks.
It is at this point one of my more shy teammates realizes that I know inequalities rather well, and asks me to explain my solutions, which I am happy to do.

Next in our lecture is convexity, and the instructor discusses convexity, concavity, Jensen, and Bernoulli. We are given quite a while to derive Bernoulli from convexity, and I do so. Sometime during this period I ask my neighbor what 微積分 means, and embarrassed to find out it means ``calculus''. Someone here is clearly an American.

At this point I thought that our lecture would end, but as I find out the cryptic entries on our schedule just mean ``lecture continued'', and hence the lecture continues. For our next game we are invited to construct multiplications tables for groups $G$, where $\left\lvert G \right\rvert \le 7$. Our goal is to provide a (consistent) partial multiplication table which other teams cannot fill in completely.
Of course, the only group of order $7$ is $\ZZ_7$, and so for my team I concoct a table for $D_6$ (seeing as I am familiar with group theory). At the end, we had just the following:
\[
  \begin{array}{r|cccccc}
    & a&b&c&d&e&f \\ \hline
    a& &e&&&& \\
    b& f&&&&& \\
    c& &&d&&& \\
    d& &&&&& \\
    e& &&&&& \\
    f& &&&&&
  \end{array}
\]
This wins; no group is able to complete the table.

Finally, our last activity for the day is a simple game: each team is asked to select ten real numbers $a_0, \dots, a_9$. Then ten coin flips $x_0, \dots, x_9$ take place, so that $x_i \in \{0,1\}$ with equal probability. Finally, the team which minimizes $\sum_{i=0}^9 \left( a_i-x_i \right)^2$ wins.
The winning team does so by selecting only $0$ and $1$, and correctly predicting nine coin flips.

Over lunch I continue writing my letter to Jessica.

The combinatorics lecture is a straight lecture -- the professor has some slides filled with problems and solutions (some of them in English, as they were copy-pasted) which he proceeds to explain one by one to us. Along the way are scattered anecdotes about teaching and coordination at the IMO; I wish I had written these comments down as they were probably pretty useful later on. Like the geometry lecture, I did not really get to work on problems, but I try to make the most of the time by post-contest analysis on the solutions (``how would I have thought of this''?). The instructor does this part pretty well too.

The second half of the combinatorics lecture is about Catalan numbers, and for this part I don't pay much attention as I've seen this countless times.
Quiz 2 takes place later that evening and I retire to bed.

We have dinner at a restaurant (or was this on the 30th? I can't remember) where I order something I think is noodles, while people discuss miscellaneous things.


\subsection{March 30}
I'm quite curious as to what will happen after breakfast, as there is no lecture to continue this time. It turns out to be pretty unstructured. We're given a set of three problems from some number of years ago, and we work on them in whatever way we like. (Solutions are given out later that evening). The second such scheduled block proceeds similarly.

The Number Theory lecture was kind of disappointing, as it was basically a straight lecture on quadratic reciprocity, which I've seen before. After reading through the handout and figuring out the main ideas of all the proofs, I find a nice proof of quadratic reciprocity online.

After Quiz 3 and dinner, we basically have free time. I think it's at this point I was talking with some people about harmonic stuff, and I end up teaching projective transformations to a small crowd of students. Lots of fun, and suddenly I become a lot more popular as people realize I actually know some math. I also help with SOS-ing an inequality. If my memory serves me correctly, it's also on this day that I start adding a bunch of the Taiwan students on Facebook.

Apparently someone has read my bary article before and is rather surprised to learn that I wrote it. Good to see that my work is being propagated!

\subsection{March 31}
Breakfast takes place in a separate building which is actually a dining hall. During this time, someone asks me if I know the solution to a geometry problem I complex-bashed. I remember this distinctly because it was the first time someone actually asked me about math outside of the class. I happily explain the procedure for it.

The mock olympiad takes place in a room not far from the one we had before, but the desks are larger and two to a person. I am not allowed to leave early after finishing the 4.5-hour exam in half an hour.
(Ironically, the geometry on the mock olympiad is also complex-bashable by the same means I described earlier that morning.)

Until dinner, the remainder of the day is a free day in the testing room. I don't do much for the first couple hours of it (which I now highly regret), while some of the other Taiwan students are doing inequalities.
Eventually I step in and do the Shortlist A2 that someone had put up. Another student asks me to explain/motivate my solution, and I happily oblige.

At this point I start giving people some American inequalities, namely TSTST 2012 \#6 and my ELMO inequality
\[ a^a b^b c^c \ge 1 \quad \text{for all} \quad a+b+c = \sqrt[7]{a} + \sqrt[7]{b} + \sqrt[7]{c}. \]
The latter was really very warmly received. I think students were later sharing that problem with each other on Facebook. Finally, as parting gift before dinner, I share with them my favorite troll problem:
\begin{quote}
  試球所有的正整數 $x$, $y$, $z$ 使得 $xy(x^2+y^2)=2z^4$.
\end{quote}
I quite enjoy their reaction upon seeing the solution.

I had a difficult time ordering dinner as I cannot read the characters. Nonetheless, I end up getting something edible. (One of the other students joked that apparently inequalities were easier than Chinese. I heartily agree.)

\subsection{April 1}
I was awoken very early this morning by a lightning storm. During breakfast I found out that only two other people had noticed the storm; the other contestants were sounds asleep.

After the second day of the mock olympiad, we are again left to our own devices. I spend a while trying to figure out the synthetic solution to the geometry problem, but I end up showing the barycentric solution over a couple blackboards. I was also called in by the graders to explain my solution to \#4 (as I made a typo which evidently caused significant confusion). There is not as much math done otherwise, mostly talking. Some of the students are disappointed by their performance on Day II as they feel they have not qualified for the second stage.

Over the last few hours I play osu and Unblock Me with some friends. The camp concludes at 5PM, and I am sad to see them go, as only one-third of the contestants will return for the third stage. I deeply hope that I am one of them.

% lightning-ing

\section{Intermission}
I learn that Facebook is very popular among teenagers in Taiwan as the main form of communication (in the US, we often use Gmail as well).
Indeed, I keep in touch with a lot of my closer friends, as well as some people that I just added, over the next several weeks.
I am very happy to get a lot of problem discussion during these chats.

Around this time I write a couple handouts, partially to practice my Chinese but also to discuss things that I was surprised to find that some of the Taiwan students didn't know (namely: complex numbers in geometry and inequalities).
The links are here\footnote{Editor note 20 Sept 2023: edited these links to the modern
  versions since the Dropbox links are no longer active}:
\begin{itemize}
  \ii \url{https://web.evanchen.cc/handouts/Ineq/Ineq.pdf}
  \ii \url{https://web.evanchen.cc/handouts/cmplx/cmplx.pdf}
\end{itemize}
There were plenty of errors in Chinese that gradually got pointed out to me. As I liked to say, 中文比不等式難.

\section{Third Camp}
\subsection{April 25}
It is good to be back. It's quite lively in the lobby and I'm happy to see everyone again.
As always, people are talking math.
Some people ask me about the handouts that I posted on Facebook, which was pleasantly surprising.

Today we begin with a number theory lecture. It covers Bernoulli numbers/polynomials and the Riemann zeta function at a rather quick pace, drawing freely on derivatives to aid with the proceedings. I manage to keep up until the last half hour, at which point the lecturer finishes up some loose ends from the second camp which I did not attend. At this point multinomial coefficients appear in a multi-variable Riemann zeta function accompanied with double integrals, and I get totally lost. I do get the impression that I outlasted almost everyone else though.

\subsection{April 26}
After breakfast in the morning we immediately take the first quiz. Both problems were fairly troll, so the room is quite noisy shortly thereafter.

In the inequalities lecture, I am amused to see that our handout is none other than Kiran Kedlaya's massive $A>B$ handout.
The lecture is pretty loosely structured, and we don't actually use the handout very much. The instructor first starts by talking about the most recent IMO inequality, 2012 \#2. He mentions that this inequality was bad for Taiwan because the students were in between ``only knowing AM-GM'' and ``able to solve almost any inequality'', which were the optimal places to be. I think this is pretty true, and I remember the scores for that problem in Taiwan were not very high.
Then we end up discussing Lagrange Multipliers, and trying to LM the IMO inequality before eventually getting bored of it.

Afterwards we actually look at the handout and spend far too much time discussing the USAMO ``log concave'' problem from the 90's. I don't think we managed to do much else during the lecture time.

The algebra lecture is taught by the same instructor I had met at the first camp, This time it's mostly concerned about linear algebra. The first $\frac 23$ of the lecture is very slow: we define a vector field and basis and so on. I almost expect an explanation of a $\QQ$-basis of $\RR$, but it never arrives as the slides then deviate into AC and ZFC and the like, as well as discussion of the ordinals and cardinals. At least I figure out that basis sums must be finite linear combinations. I also run into the following really cool problem while we discuss ZFC.
\begin{quote}
  Infinitely people in a room are wearing red or blue hats such that they can see everyone else's hat but not their own.
  Simultaneously they all guess the color of their hat.
  Prove that it's possible for the participants to devise a strategy so that at most finitely many people get their color wrong.
\end{quote}
I definitely now have a much better appreciation for axiomatic set theory.

The last $\frac 13$ is more interesting. We discuss a few olympiad-style problems (back to Earth again!) which can be solved by linear algebra. At this point combinatorics show up, and I find that I am unable to read some of the slides, and we have a good laugh as people try to translate problem statements for me into broken English. 中文比不等式難。

You'll notice I've used thirds to refer to a $2$-hour lecture, but this actually makes sense because the lecture went one hour overtime. I had thought that the schedule for this camp was nice in that it was much harder for the lectures to go overtime (unlike last camp), but this ended up not being true, as we simply went through the time slot allocated for Quiz 2. Hence we begin Quiz 2 an hour late.

Afterwards is a brief dinner starting at 7PM, while people discuss the Quiz 2 problems. The session ends an hour early at 8PM, and we retire to the dorm. I go to sleep shortly, but I heard a lot of the others stayed up trying to finish Quiz 2.

\subsection{April 27}
The morning begins with a combinatorics lecture from Dr.\ Yeh (if I recall correctly, the same one as last time). The format is identical; we talk about several IMO problems from slides, but somehow this one felt a lot more\dots organized?\dots than the previous one.
While I remember feeling like I was just reading solutions this time, the solution commentary here was actually really insightful, spending a lot of time talking about how to come up with a solution and what the IMO leaders thought about the problems and so on.
I also actually felt like I understood everything in this lecture (last time there were some bits and pieces I missed).
I think the only complaint I have is that we spent too much time on the IMO \#1's at the beginning; they were really very easy problems and I think anyone in the room could have reliably gotten them on any day.

I'm pretty exhausted by the end of the geometry lecture, again the same instructor as last time, and roughly the same format -- we work on handout problems while the lecturer reads out solutions. I think the problems were ``themed'' as transformation problems, but I didn't notice this terribly much.

Although I felt after these lectures that I didn't really get a chance to do much attempting problems (particularly in the NT and inequality lectures), I did feel that the lectures in the third camp were better in this regard.

After these two lectures we have lunch, during which people begin to be called to the grading room. During this time I work on a ``problem'' from CBD that I heard earlier this morning:
\begin{quote}
  Is it possible to arrange the polynominoes of size $7$ into a rectangle?
\end{quote}
This is about as good a problem as my $xy(x^2+y^2)=2z^4$.

We then take the third quiz; it contains a geometric inequality that no one solves.
Posting the contest there's a lot of complaining about how there's no traditional geometry on the quizzes so far, and a lot of people seem to have gotten no problems at all, which is probably really frustrating.
Actually I think not that many people got a positive score on this quiz, either. CBD mentions that he's seen \#1 before somewhere but didn't bother doing it or reading the solution.

For the practice sessions, one of the previous Taiwan IMO contestants comes and gives us four geometry problems. They are all high-numbered geometry problems from the IMO Shortlist. While I solve the first one quickly because I have seen the main idea before, I am hopelessly stuck on the other three. In fact, the last problem is the IMO Shortlist 2012 G8 that I had been repeatedly complex bashing. So I am extremely impressed when the students collaboration is enough to solve the second problem, and the instructor seems to be able to solve that G8 on the spot at the white board (it was clear he was simply solving the problem rather than remembering a solution). Perhaps I've just gotten worse at geometry.

We have dinner out again, at the same place as the fourth day of first camp (I just remember this by the spaghetti they served) and then get back to the practice session, where we keep discussing random geometry problems. I'm disheartened to find that it seems I can no longer solve synthetic geometry, well, synthetically. But alas! I have complex numbers.

On a dinner note, show that it's possible to go from any two of the following figures to the other by making one cut to obtain two pieces, and rearranging them.
\begin{figure}[ht]
  \centering
  \begin{asy}
    size(8cm);
    pen p = 1.6 * grey;
    pen b = black;
    filldraw(scale(3)*unitsquare, p, b);
    filldraw(shift(1,1)*unitsquare, white, b);
    label("$A$", (1.5,0), dir(-90));
    add(shift((-5,0))*CC());
    filldraw(scale(3)*unitsquare, p, b);
    filldraw(shift(1.5,1.5-0.707)*rotate(45)*unitsquare, white, b);
    label("$B$", (1.5,0), dir(-90));
    add(shift((-5,0))*CC());
    filldraw((0,0)--(3,0)--(3,2)--(2,2)--(2,3)--(0,3)--cycle, p, b);
    label("$C$", (1.5,0), dir(-90));
  \end{asy}
\end{figure}

\subsection{April 28}
In the morning we head over to the H building and have breakfast in the lobby, before heading up to the highest floor to take the first TST.
I make the mistake of not refilling my water bottle before this happens, and thus get somewhat thirsty during the test.
I also make the mistake of not turning off my phone, which goes off in the last few minutes of the exam.
Fortunately I'm not the only one making mistakes today, as there are some pretty clear typos in \#1.

After the test concludes at the usual 1PM, we find out that once again the test was too difficult -- the numbers of solved problems seems between $0$ and $1$ for everyone, and I don't think anyone nailed \#3.
At this time Cheng-Der Fuh calls over the students one by one to chat briefly. At first I think this is the 口試 but am later told that it has not come.
My chat lasts only briefly, but it's a fairly pleasant conversation as I haven't had much of a chance to actually talk to Dr.\ Fuh despite having met him so many times.

As people are called over to the grading room, most of the contestants start playing games on tablet devices (to my surprise, as I had expected them to start cracking \#3, which they only do later that evening). Eventually I get into it as well, pulling out {osu!} on my own device.
Anyways it's interesting to see that some of the games which are popular here (Flow, anyone?) are also the same ones popular in the States.
There are also some ones I haven't seen though (Ceramic Destroyer).

I get called over to quickly clarify the ending of my \#2 solution, but I'm told it's mostly because I had written ``為什麼數學這麼難'' on my question sheet (提問單). I tell this to the rest of the contestants after they ask me how the coordination went and we have a good laugh.

After dinner I'm pretty exhausted and mostly want to sleep, but the rest of the students seem adamant now on solving the third problem with hints from the so-called TS (presumably one of the past IMO contestants). They eventually do after using some properties of the mixtilinear incircle that I again did not know existed. Perhaps I should expand the section on mixtilinear incircles in my book, I keep finding things that I had no idea were true!

I'm really very sad this is the last full day. I'm going to miss everyone from camp a lot when I get home.

\subsection{April 29}
Today is the last day. We take the mock IMO as usual from 8:30AM-1PM. The last four hours of camp take place in the same classroom, as the grading proceeds and students are occasionally called over to the grading room.

Everyone seems really anxious regarding the IMO results, modulo one or two people who either feel they have no chance or feel they have almost certainly made it. I personally should have been in the second category had I solved the geometry problem today (meaning I more or less majorized everyone's scores), but I nonetheless feel somewhat nervous as well. The tests have been sufficiently hard that the scores are all within one or two problems of each other.

We hang around for a few hours before four contestants are called down for the oral exam. No one is sure why these particular four students were selected at first. Some time after these four return, all six of the remaining contestants are sent down as well.

My oral exam is more an interview than anything else. All the teachers and instructors are there, and they mostly ask questions about how the training in the US and the training in Taiwan are different. I'm happy to respond, I mostly echo what I had already written earlier in the report.
The teacher also why I have been using complex numbers so much lately, to my mild embarrassment, and recommends that I practice my synthetic geometry when I get a chance. This is really ironic as I had been deliberately \emph{not} practicing my synthetic geometry earlier this year since I felt I needed far more work on other subjects. But after all the geometry problems I missed at the third camp, I think the suggestion is well-taken. I hope that my synthetic geometry abilities return soon, as I do miss them a lot.

Shortly afterwards the camp concludes, and I prepare for the USAMO that night, which I would take from 12:30AM - 5AM as required by time zones. I say a final goodbye to everyone, knowing full well that I will possibly never see some of the others again. Thank goodness for Facebook, though!

Two days later I learned that I would be participating in the 55th IMO as contestant TWN2. I had a really enjoyable experience at the IMO selection camps, and want to thank everyone that made these events possible.

\chapter{Contest Analysis}
\section{Stage 1 Contests}
\subsection{Quiz 1}
\begin{enumerate}
  \ii 已知 $a$, $b$, $c$ 為正數,試證不等式 \[ 3(a+b+c) \ge 8\sqrt[3]{abc} + \sqrt[3]{\frac{a^3+b^3+c^3}{3}}. \]
  \ii 是否能找到十個集合 $A_1$, $A_2$, \dots, $A_{10}$ 同時滿足下列條件:
  \begin{enumerate}[(i)]
    \ii 每個集合有三個元素 $\left\{ a,b,c \right\}$, 其中 $a \in \left\{ 1,2,3 \right\}$, $b \in \left\{ 4,5,6 \right\}$, $c \in \left\{ 7,8,9 \right\}$.
    \ii 任兩集合都不相等。
    \ii 將這十個集合依次一圈 ($A_1$, $A_2$, \dots, $A_{10}$), 則任意相鄰的兩集合沒有共同元素,但是任意不相鄰的兩集合都有共同元素。(註. $A_{10}$ 與 $A_1$ 相鄰。)
  \end{enumerate}
\end{enumerate}
\#1 was an inequality, which succumbed pretty quickly to power mean and which I completed in five minutes. Many of the Taiwanese students failed to solve this problem. This gave me a much better appreciation of the disparity in which the US trains inequalities compared to other countries. Unfortunately for the USA and fortunately at Taiwan, inequalities are far out of fashion at the IMO (for now).

\#2 was a pretty troll combinatorics problem -- it turns out that the answer is ``yes'' and an explicit construction exists. Somehow I managed to fake-solve this problem. I think the hardest part of this problem was realizing the correct answer; in hindsight, the small numbers should have been a tip-off. I did not solve this in the 80 minutes I had; this turned out to be the only problem I missed.
The take-away lesson from this problem is to never be too confident in yes/no problems. I likely could have solved this if I had spent all 80 minutes looking for a counterexample, and indeed a lot of the other students found the correct answer ``by accident'' as they were playing around looking for ways to prove the answer was ``no''.

\subsection{Quiz 2}
\begin{enumerate}
  \ii 試求所有的函數 $f : \NN_0 \to \ZZ$ 滿足
  \[ f(2) = 7, \quad f(mn) = f(m) + f(n) + f(m)f(n), \text{ 對所有 $m,n \in \NN_0$}. \]
  \ii 設一個三角形的三邊長分別為 $a$, $b$, $c$ 而 $a$, $b$, $c$ 三邊所對應的高分別為 $h_a$, $h_b$, $h_c$。證明 $\left( \frac{a}{h_a} \right)^2 + \left( \frac{b}{h_b} \right)^2 + \left( \frac{c}{h_c} \right)^2 \ge 4$.
\end{enumerate}
I began with \#2, visibly a very easy inequality if you know the formula $16[ABC]^2 = \left( a^2+b^2+c^2 \right)^2 - 2(a^4+b^4+c^4)$. Again, I was surprised how many students did not solve this cleanly. US students really are very strong at inequalities.

I returned to \#1, a functional equation. I pretty quickly discovered that it was essentially just solving $f(mn) = f(m)f(n)$ for integers $m$, $n$, along with the restraint that $f$ is increasing. This is fairly routine integer approximation for anyone that's seen it before, which fortunately included me. I solved this in half an hour and spent remaining time doodling with APMO 2014 \#5.

\subsection{Quiz 3}
\begin{enumerate}
  \ii 設圓 $O_1$, $O_2$ 的半徑分別為 $R_1$, $R_2$, 且此兩圓交於 $A$, $D$ 兩點。過 $D$ 作一直線 $L$, 設 $L$ 分別再交圓 $O_1$, $O_2$ 與 $B$, $C$ 兩點。
  現在讓兩圓圓心的距離可以變動,直線 $L$ 也可以變動。當 $\triangle ABC$ 的面積達到最大時,求 $AD$ 的長度。
  \ii 令 $n$ 為正整數。求最小的正整數 $k$, 使得:若 $a_1$, $a_2$, \dots, $a_d$ 滿足 $a_1 + a_2 + \dots + a_d = n$,且對於所有 $i=1,2,\dots,d$, 都有 $0 \le a_i \le 1$, 則我們可以將這 $d$ 個數分成 $k$ 組(其中若干組可為空集合),使得每一組的數字總和至多為 $1$.  % ISL C1
\end{enumerate}
\#1 was geometry at last, so I started by sieging that. It turned out to just be the spiral similarity configuration that I had seen a lot in Yufei. Having missed that once on USAMO 2013 \#6, I was not about to miss it again. The problem fell after just a few minutes.

Problem 2 was a combinatorics min/max problem, so I was deathly scared that I would fail to solve it. Luckily, after examining a few small cases I was able to correctly guess the equality case. Then I just had to act greedily to show the equality case was tight. This surprised me as I usually fail to solve combinatorics problem, particularly those that look at local structure. I concluded that looking at the equality case was a good thing to be doing.

\subsection{TST I, Day 1}
\begin{enumerate}
  \ii 設 $f(x) = x^n + a_{n-2} x^{n-2} + a_{n-3}x^{n-3} + \dots + a_1 x + a_0$ 為實係數 $n$ 次多項式 ($n \ge 2$). 如果 $f(x) = 0$ 的根都是實根,試證每一根的絕對值都小於或等於 $\sqrt{\frac{2(1-n)}{n} a_{n-2}}$。
  \ii 給定正整數 $k$, 試球所有整係數多項式 $f(x)$, 使得對於所有正整數 $n$ 都有 $f(n)$ 整除 $(n!)^k$, 此處 $n! = 1 \cdot 2 \cdot \ldots \cdot n$.
  \ii 設 $\triangle ABC$ 的內切圓圓心為 $I$, 且該內切圓分別為 $CA$, $AB$ 邊切於點 $E$, $F$. 令點 $E$, $F$ 對 $I$ 的對稱點分別為 $G$, $H$.
  設點 $Q$ 為 $GH$ 與 $BC$ 的交點,並設點 $M$ 為 $BC$ 的中點。證明 $IQ$ 與 $IM$ 垂直。
\end{enumerate}
I saw that \#1 was an inequality (phrased with Vieta), \#2 was polynomial number theory, and \#3 was Euclidean geometry. Guess which one I started with?

After looking at the geometry for about 30 seconds I that complex numbers would prove lethal. Indeed, $Q = DD \cap EF$, $B$ and $C$ were tangents to the incircle, and $m = \half (b+c)$. The final condition $\angle QIM = 90\dg$ was also perfect. I applied the computations and the problem was instantly destroyed. I was fortunate to know the chord intersection formula $\frac{ab(c+d)-cd(a+b)}{ab-cd}$, which the other students evidently did not.

I proceeded back to problem 1, the inequality, which I was confident I could solve. Initially I misread the problem and confused myself, but soon realized my error. Afterwards the path to the solution became clear: giving reals with sum $0$ and sum of squares $a$, place a bound on the maximum possible absolute value. (The rest of the polynomial is evidently irrelevant at this point.) This could only be Cauchy (or QM-AM with a tiny bit more work), and indeed it was.

Thus in less than 15 minutes I had already completed two problems. Knowing I was in good shape, I proceeded to begin on \#2. I actually thought this problem was both nice and not so easy ({i.e.} I don't think I could have solved it reliably), but I was having a good day and noticed the key lemma right away. It was inspired by considering small cases, where I discovered $f(2) \mid 2^k$, $f(3) \mid 3^k$ (as $f(3)$ is odd) and so on. I then asserted that for any prime $p$ and $n$ with $p \nmid n$ we had $p \nmid f(n)$. After that it was easy to obtain $f(p) \mid p^k$ for every prime $p$; the rest was just bounding. Other students found other ways to finish.

Overall I thought this day was very easy, and was rather irritated that I did not gain any ground from solving all three problems in half an hour.  I was not allowed to turn in the exam early. Over the next four hours the other students caught up to me, while I stared absently into space and/or tried expanding some inequalities I remembered. So despite a performance I was rather proud of myself for, I had not gained any ground. I prayed that the future exams would be equally fortunate, but based on the 2012 and 2013 exams I felt that my good luck was likely wasted.

\subsection{TST I, Day 2}
\begin{enumerate}
  \setcounter{enumi}{3}
  \ii 在 $\triangle ABC$ 中,設點 $D$ 在 $BC$ 邊上且 $AD$ 平分 $\angle BAC$, 並設 $AD$ 的中點為 $M$。
  設以 $AC$ 為直徑 $\omega_1$ 與 $BM$ 交於點 $E$, 以 $AB$ 為直徑的圓 $\omega_2$ 與 $CM$ 交於點 $F$。
  證明 $B$, $E$, $F$, $C$ 四點共圓。
  \ii 證明: 存在無窮多正整數 $n$, 使得 $n^4+n^2+1$ 的最大質因數, 和 $(n+1)^4+(n+1)^2+1$ 的最大質因數相同。 % ISL N3
  \ii 某國有數個城市,其中若干個城市之間有航線相連;航線都是雙向的。已知從該國中任選兩個城市,都可以從其中一個城市,透過一系列航線抵達另一個城市。
  定義兩個城市的距離為從一個城市抵達另一個城市所需的最小航線數量。已知對於任何一個城市,至多都只有 $100$ 個城市與其距離恰為 $3$。
  試證:不存在一個城市,有超過 $2550$ 個其他城市與其距離恰為 $4$。% ISL C6
\end{enumerate}
The topics today: \#4 was geometry, \#5 was number theory, and \#6 was the dreaded combinatorics, graph theory in fact.

The \#4 was really very hard, despite its place on the test. If I were to speculate I would have thought it was a misplaced shortlist problem. I looked a while for synthetic solutions but did not see any way to proceed. However, I also saw that within half an hour this problem could be bashed with barycentric coordinates\footnote{Or, as the Taiwan students called it, 用重心座標來炸. Interesting how every country has its own well-established term for bashing.}. I decided to leave it for now, knowing that if worst came to worst I could just bary it.

I moved on to \#5, a problem on largest prime factors. It actually took me a while to notice that $n^4+n^2+1=(n^2+n+1)(n^2-n+1)$, but afterwards the problem became very similar to a Russian problem I had done at MOP 2012, and hence I solved this swiftly thereafter. The main idea is to just show that the largest prime factor of $n^2+n+1$, which we'll call $g(n)$, cannot keep strictly increasing.  But $g(n^2) = \max \left\{ g(n), g(n-1) \right\}$, and hence we're done. (I actually noticed this by factoring $n^2+n+1$ for $n=1,2,\dots,16$.) I was told that other students used the Prime Number Theorem, which I confess to not being familiar with.

Finally the dreaded \#6 -- except I solved it within an hour. It looks impenetrable at a glance, and I think the hardest part is getting a foothold in the problem to start. The way I solved this was by looking for the equality case, and then thinking hard about \emph{why} it was the equality case. Once I did, I made a crucial claim about a structure I called a beanstalk\footnote{Which consists of a node a distance $1$ from a root followed by paths to a bunch of distance $4$-nodes, and with all beanstalks disjoint. The claim is that each ``root'' of a beanstalk is a distance $3$ from some node in every other beanstalk (as well as of course all the far nodes inside the beanstalk, called beans). This gives the desired bound.} It took a lot of work to clean up and make rigorous, but it worked. I gained a much better appreciation for equality cases after I finished writing it up.

At last I returned to \#5, and after half an hour of computation (woah this quadratic factors?) I finished the proof. I was actually not sure if I would finish, as I usually never solve quadratics when bashing, but I managed to pull through this time by back-solving in order to figure out what the common root had to be, and then factoring out the quadratic using Vieta. It took embarrassingly long, actually, but I finished.

I finished the exam with two hours to spare, which I spent checking my \#6 (which ended up being correct anyways) and looking for a synthetic solution to \#4 (which I failed at).

\section{Stage 2 Contests}
\subsection{Quiz 1}
\begin{enumerate}
  \ii 令 $n$ 為以正整數且令 $a_1$, \dots, $a_{n-1}$ 為任意實數。定義數列 $u_0$, \dots, $u_n$ 與 $v_0$, \dots, $v_n$ 如下:
  \begin{align*}
    u_0 = u_1 = v_0 = v_1 &= 1, \\
    u_{k+1} &= u_k + a_ku_{k-1}, \\
    v_{k+1} &= v_k + a_{n-k} v_{k-1}
    \text{ 對於 $k=1,\dots,n-1$}
    .
  \end{align*}
  試證: $u_n = v_n$. % ISL A1
  \ii 設 $ABCDEF$ 為凸六邊形,其中 $AB=DE$, $BC=EF$, $CD=FA$, 並且 $\angle A - \angle D = \angle C - \angle F = \angle E - \angle B$。 證明對角線 $AD$, $BE$ 與 $CF$ 共點。 % ISL G5
\end{enumerate}
I recognized the first problem as very similar to IMO Shortlist 2009 Problem C3. (In full disclosure I mentioned this in my solution). Nevertheless, it took me about an hour to hack together the necessary ideas in order to solve the problem. While the ISL problem limited the $\epsilon_i$ to $\{0,1\}$ for its variables, I had to solve the problem for any real numbers. These are actually equivalent since the polynomials are multilinear (combonull, anyone?), and I chose the set of variables $\{-\frac 14, 0\}$ in order to force my bijection to work. I am very interested in knowing the actual solution.

The second problem was a geometry problem, but it was one of those with a strange condition, namely $\angle A - \angle D = \angle B - \angle E = \angle C - \angle F$. I flailed around with this for a while but did not get very far. I should find the solution.

\subsection{Quiz 2}
\begin{enumerate}
  \ii 設 $\triangle ABC$ 的內心與外心分別為 $I$ 與 $O$。
  作直線 $L$ 使與 $BC$ 邊平形,並與 $\triangle ABC$ 的內切圓相切。設 $L$ 與 $IO$ 交於 $X$ 點,另取 $L$ 上的一點 $Y$ 使得 $YI$ 垂直於 $IO$。
  證明 $A$, $X$, $O$, $Y$ 四點共圓。
  \ii 設 $r$ 為一正整數, 而 $a_0$, $a_1$, \dots, 為無窮多個實數所成的序列。
  假設對與任意的非負整數 $m$ 和 $s$, 都存在正整數 $n \in [m+1, m+r]$ 使得
  \[ a_m + a_{m+1} + \dots + a_{m+s} = a_n + a_{n+1} + \dots + a_{n+s}. \]
  試證:存在 $p \ge 1$, 使得對於所有非負整數 $n$, $a_{n+p} = a_n$。 % ISL C5
\end{enumerate}
The first problem was a traditional Euclidean geometry problem, but I was not able to solve it synthetically. Nonetheless, I pulled off a complex number bash in about half an hour. In retrospect I'm quite surprised I succeeded, as I tried to reproduce the computation later and definitely spent far more than two hours making mistakes. I must have been having a good day.

Like the first quiz, I did not make much progress on the second problem, but I hoped for a $1$ nonetheless. In the last few moments I put together what I thought was a tricky inductive solution, but later found a rather significant hole. I will have to ask for the solution later.

\subsection{Quiz 3}
\begin{enumerate}
  \ii 令 $a_i > 0$,  $i = 1,2,\dots,n$, $\sum_{i=1}^n a_i = 1$.
  試證: 對任意正整數 $k$,
  \[ \left( a_1^k + \frac{1}{a_1^k} \right) \left( a_2^k + \frac{1}{a_2^k} \right)\left( a_n^k + \frac{1}{a_n^k} \right) \ge \left( n^k + \frac{1}{n^k} \right)^n. \]
  \ii 試求函數 $f : \QQ \to \ZZ$, 滿足
  \[ f\left( \frac{f(x)+a}{b} \right) = f\left( \frac{x+a}{b} \right) \]
  對於所有 $x \in \QQ$, $a \in \ZZ$ 和 $b \in \NN$ 都成立。 % ISL N6
\end{enumerate}
The first problem was an $n$-variable inequality which could be rewritten in the form $\sum \ln f(x_i) \ge \text{constant}$. This succumbed to Jensen rather easily, much to my surprise. (I was fully prepared to zap it with $n-1$ EV).

The second problem was a rather sadistic functional equation $f\colon \QQ \to \ZZ$ which had both the constant solution as well as $f(x) = \left\lfloor x \right\rfloor$ as a solution. I don't remember the details of what I tried, but I managed to pin down the function in lots of places after handling the constant solution. Unfortunately, I ran out of time before I could finish. I think another half hour would have done the trick, but hoping for a $2$ here.

\subsection{TST II Day 1}
\begin{enumerate}
  \ii 設 $\omega$ 為三角形 $ABC$ 的外接圓。 令 $AB$ 邊的中點為 $M$, $AC$ 邊的中點為 $N$, 並令 $\omega$ 上不含點的 $BC$ 弧的中點為 $T$。
  設三角形 $AMT$ 的與 $AC$ 中垂線交於 $X$ 點, 三角形 $ANT$ 的外接圓與 $AB$ 邊的中垂線交於 $Y$ 點;假設 $X$, $Y$ 兩點皆位於三角形 $ABC$ 的內部。
  直線 $MN$ 與 $XY$ 交於 $K$ 點。 證明 $KA=KT$。 % ISL G2
  \ii 試求所有的函數 $f\colon \ZZ_{\ge 0} \to \ZZ_{\ge 0}$ 滿足
  \[ f(f(f(n))) = f(n+1)+1, \text{ 對所有的非負整數 $n$ 皆成立。} \]
  % ISL A5
  \ii 給定一個大於 $1$ 的正整數 $k$。 甲、乙兩人玩以下的數字遊戲:在遊戲開始時,有一個正整數 $n \ge k$ 被寫在黑板上。接著,從甲開始,兩人輪流有進行以下動作:擦掉寫在黑板的數 $m$, 並在黑板上寫個與 $m$ 互質的正整數 $m'$ 且 $k \le m' < m$。
  第一個無法寫下數字的人輸。 \\
  對於在黑板上的數字 $n \ge k$,如果乙有必勝法,則稱 $n$ 是個好數字; 反之, $n$ 是個壞數字。 \\
  現在, 假設 $n,n' \ge k$, 且質數 $p \le k$ 整除 $n$ 若且唯若 $p$ 整除 $n'$。試證: $n$ 和 $n'$ 要不同時是好數字,要不同時是壞數字。 % ISL N5
\end{enumerate}
The subject distribution for that day was GAN, although the last problem was arguably actually combinatorics with NT as icing. It was game theory, for heaven's sake. Unfortunately it was definitely counted as N as I faced a C the next day without an N to fall back on.

The hardest part of the opening geometry problem was to get the diagram right. After this there's a rather conspicuous claim you can make; I don't remember the point names, but basically the problem asked to show three lines are concurrent -- let's call them $\ell_1$, $\ell_2$ and $\ell$ -- and you do this simply by proving that $\ell_2$ is the reflection of $\ell_1$ over $\ell$, from which concurrence is obvious. Angle chasing gives that the two points which determine $\ell_2$ are reflections of two of the points on $\ell_1$, making this straightforward. Incidentally one of the points I added into the figure (midpoint of $\ol{AD}$ if I recall correctly) happened to be the center of a circle which passed through all four of the points, making this claim comfortable to make. The total solve time here was maybe 30 to 60 minutes.

Problem 2 was a functional equation which I will be hard pressed to forget: $f^3(n)=f(n+1)+1$ where $f$ takes $\ZZ_{\ge 0}$ to itself. I tried this for a while and made some reasonable progress; in particular $f$ was injective, mostly surjective, et cetera.
I also tried my usual trick of $f^4(n) = f(f(n+1)+1) = f(f(n)+1)+1$, but didn't see where to go after that.
So I put this down as started Problem 3.

This was one of those problems that seems intuitively clear but is difficult to prove. I think it would have been much easier if we had some advances in number theory (like the twin prime conjecture, or something) but alas! This was standard game theory strategy, but the multiplicative structure of the primes stood in the way.

I tried making some bold claims (e.g. we only need the primes $p$ with $p \mid k$) which turned out to be false. It took me almost two hours of scouting examples before I made the right claim. Namely, define a number to be \emph{pure} if it has only prime factors at most $k$, and call the \emph{type} of a number the set of prime factors $\le k$ it has. Then the smallest integer $> k$ of any given type is pure. This claim was trivial to prove, but it gave me the foothold necessary in order to complete an induction. The details nonetheless took a couple pages.

Finally I returned to Problem 2 with around 90 minutes left, shooting for the $777$ that day. I had the idea of letting $t = f(f(0)+1)+1$ so that
$f(f(1)+1) = t+1$. I then realized this in turn gave $f(f(2)+1) = (t+1)+1$, and in general I now had
\[ f\left( f(n)+1 \right) = t+n \]
for any nonnegative integer $n$.

I sensed that I was almost done, but none of the things I tried for the next half hour or so yielded much further progress. Finally, I realized the origin of the above equation was actually also
\[ f^4(n) = f(f(n)+1) = n+t. \]
I suddenly realized I was being an idiot, as I could simply apply my composition trick again to get
\[ f^5(n) = f(n+t) = f(n) + t. \]
There were only about 15 minutes or so left on the clock by the time I got to here. I grit my teeth and tried to complete the solution, praying that I could pin the problem down within the next few minutes.
Unfortunately, in the last 5 minutes or so I realized why I couldn't get the foothold I needed: another solution existed.
Somehow I managed to convince myself it was
\[ f(n) =
  \begin{cases}
    n-1 & \text{$n$ odd} \\
    n+1 & \text{$n$ even}.
  \end{cases} \]
It was in fact something slightly more complicated with modulo $4$. Unwittingly I estimated a $5$ for that problem, but I think $2$ is closer now. I really think another half hour would have done this problem in though.

Which is not to say that the other students did well. I later learned that almost everyone had spent their time getting trolled by \#2, that no one had even realized there was a pathological solution (at least I tried to construct one, which is probably worth points), and as a result no one had really attempted the easier \#3. So I don't feel so bad about that.

\subsection{TST II, Day 2}
\begin{enumerate}
  \setcounter{enumi}{3}
  \ii 證明:在任意由相異的 $2000$ 個實數所成之集合中,存在兩對實數 $a > b$ 與 $c > d$, 其中 $a \neq c$ 或 $b \neq d$, 使得
  \[ \left\lvert \frac{a-b}{c-d} - 1  \right\rvert < \frac{1}{100000}. \] % ISL A2
  \ii 考慮一個簡單圖 $G$。我們有兩種動作:
  \begin{enumerate}[(1)]
    \ii 若 $v$ 是 $G$ 的頂角,而 $v$ 的度是奇數,可以把 $v$ 刪除。
    \ii 可把整個圖 $G$ 改成 $G \times K_2$。
  \end{enumerate}
  試證:可以達到一個圖 $H$ 使得 $H$ 都沒有邊。 % ISL C3
  \ii 設 $P$ 為三角形 $ABC$ 內一點,直線 $AP$, $BP$, $CP$ 分別與三角形 $ABC$ 的外接圓交於 $T$, $S$, $R$ 點 ($T \neq A$, $S \neq B$, $R \neq C$)。
  設 $U$ 為線段 $PT$ 內一點。過 $U$ 與 $AB$ 平行的直線分別與 $CR$ 交於 $W$, 過 $U$ 與 $AC$ 平行的直線分別與 $BS$ 交於 $V$ 點。
  最後,設過 $B$ 與 $CP$ 平行的直線, 與過 $C$ 與 $BP$ 平行的交於 $Q$ 點。 已知 $RS$ 與 $VW$ 平行, $\angle CAP = \angle BAQ$。
\end{enumerate}
The problem choices were ACG, although \#4 had a rather combinatorial flavor. (Ugh.)

I started with the G, expecting something tough seeing as it was \#6. It wasn't. The totally random selection of a point $U$ on segment $AP$ could easily be eliminated by just shifting upwards; in a sense $U = A$ was a sufficient case to examine. Then the parallel condition just reduced to $BYXC$ being cyclic, where $Y = BP \cap AC$ and $X = CP \cap AB$. The end condition was $\angle BAP = \angle CAQ$ (i.e.\ $AP$ and $AQ$ isogonal), where $Q$ was the reflection of $P$ over the midpoint of $\ol{BC}$. And this could only be barycentric coordinates\dots

Unfortunately the rest of the day did not go well. I don't think I had anything interesting in any of the problems for the next four hours. I think some of the other guys got \#5, so I was at a rather severe disadvantage after this day. I hope the third camp has less combo.

I tried a lot of bounding-style techniques on \#4 after assuming WLOG the minimum and maximum were $0$ and $1$, but the requirement of $10^{-5}$ was simply too sharp. In the last few minutes I realized that one of the bounds I was using, $(1+\eps)^n \ge 1+n\eps$, was extremely weak. (When did I get the idea Bernoulli was sharp?). Indeed, this bound is good if $\frac{1}{\eps} \gg n$, but in fact the opposite was true; I think $\eps = 10^{-5}$ but $n = 10^{8}$, and so the $(n\eps)^2$, $(n\eps)^3$ terms were actually significant. I hurriedly tried to write down the rest of the argument, hoping that it would work.
Unfortunately, while I was trying to prove an expression was greater than $1$, it turned out to be approximately $\frac 34$. Bummer; I hope I get some sympathy for getting that close, since my earlier efforts only got to $10^{-2}$ or so. I think another thirty minutes and I might have been able to squeeze the numbers to get the thing greater than $1$, but alas, $4.5$ hours is really too short!



\section{Stage 3 Contests}
\subsection{Quiz 1}
\begin{enumerate}
  \ii 給定 $6 \times 6$ 的方格,並記第一列六個方格座標為 $(1,1)$, $(1,2)$, \dots, $(1,6)$, 其類似推。
  對於任意的 $k=0,1,\dots,5$, 滿足 $i-j \equiv k \pmod 6$ 的六個格子 $(i,j)$ 稱為在同一條的 (故共有六條對角線)。
  試問:能否將 $1,2,\dots,36$ 寫在 $6 \times 6$ 的方各中,同時滿足
  \begin{enumerate}[(1)]
    \ii 每一列的和都相等。
    \ii 每一行的和都相等。
    \ii 每一條對角線的和都相等。
  \end{enumerate}
  \ii 甲、乙兩人玩以下的數字遊戲:從甲開始,兩個人輪流自 $1$ 到 $9$ 的數字中不重複的選一個數字出來,並且把選出的數字由左依序成一個七位數(級 $\ol{A_1B_2A_3B_4A_6B_6A_7}$)。如果排出來的七位數是某個個完全齊次方數的末七位數字,則甲獲勝;否則的說,乙獲勝。 \\
  請問誰有必勝策略?
\end{enumerate}
What a troll quiz. Problem 1 is another yes/no question, in which we are asked to fill a $6 \times 6$ board with the numbers $1$ to $36$ such that the sum of the columns, rows, and six diagonals (wrapping around) in the same direction are identical. Because of the unfortunate Quiz 1.2, I immediately assume the answer is yes, and play around with a few constructions. When this fails, I decide to move on to \#2 temporarily.

In my infinite wisdom I somehow miss the condition that the digits chosen in this problem are distinct. Fortunately, this doesn't change the answer. I spend a lot of time wondering about the density of the seventh powers modulo $10^7$. It becomes immediately clear to me that I should be probably looking at the ones which are coprime to $10$. This starts to lead me to think the answer is that Bob wins, and I start looking at the density of these seventh powers. Unfortunately, I end up mostly confusing myself, and end up convincing myself that everything ending in $\{1,3,7,9\}$ is a seventh power modulo $10^7$.

When I rub my eyes again, I realize that this is actually the case. Wait, can't we pick $A_7=1$ then?\footnote{You actually can't always -- again, I missed the distinct condition. But fortunately Bob moves only three times, so one always has a choice of $\{1,3,7,9\}$ left.} I rub my eyes again, check my proof another two times, and cannot find a flaw in it. Wow. What a silly problem.

I return to fiddling with constructions for \#1.
I am able to construct from four $3 \times 3$ magic squares a $6 \times 6$ square which more or less works modulo $9$. The problem then became to adjust by $9$, $18$, $27$ to complete the construction. At this point I am convinced the answer is yes (why else $6$?), since I have made so much progress. I then spend the next hour flailing around.

Post-contest, I discover the answer is ``no''; any number of the form $4k+2$ has issues modulo $2$.
Oops. I also finally realize the part about distinct digits. Fortunately, I had written ``We claim that Alice wins if she chooses $A_7 \in \left\{ 1,3,7,9 \right\}$'' as the first line of my solution, so I was safe. It turns out the person behind me (a silver medalist last year by $1$ point) had the exact same thought process on both problems, including misreading \#2, except he was less fortunate and wrote ``pick $A_7=1$'' for the first line.
I think this is probably the weirdest case of me lucking out I've seen in a while.

\subsection{Quiz 2}
\begin{enumerate}
  \ii 在凸六邊形 $ABCDEF$ 中, $AB \parallel DE$, $BC \parallel EF$, $CD \parallel FA$, 以及
  \[ AB+DE = BC+EF = CD+FA. \]
  將邊 $AB$, $BC$, $DE$, $EF$ 的中點分別記 $A_1$, $B_1$, $D_1$, $E_1$. 設點 $O$ 為線段 $A_1D_1$ 及 $B_1E_1$ 的交點。
  證明 $\angle D_1OE_1 = \half \angle DEF$.
  \ii 令 $m$ 是一不為 $0$ 的整數。試求所有的實係數多項式函數 $P(x)$ 使得
  \[ \left( x^3-mx^2+1 \right) P(x+1) + (x^3+mx^2+1)P(x-1) = 2(x^3-mx+1)P(x) \]
  對所有的實數 $x$ 均成為。 % ISL A6
\end{enumerate}
Complex numbers on \#1. It took me longer than it should have because I kept miscomputing, but eventually I manage to fix my mistakes. I think if I had been more sober, this could have taken me (and definitely would have taken Victor Wang) about 10 minutes at most, but alas I kept borking up the calculation. I also did seriously try to consider synthetic approaches, but when you have those stupid hexagons with contrived length conditions, there's really not much you can do. (After the contest I hear that Quiz 2.1 was also a complex numbers problem. Those hexagons!)

No one solves \#2, and my efforts with the Mahler differences are useless. It may be worth noting that I misread the problem several times.

\subsection{Quiz 3}
\begin{enumerate}
  \ii 正整數 $x_1$, $x_2$, \dots, $x_n$ ($n \ge 4$) 依序排列在圓周上,任意 $x_i$ 的左右鄰居之數字會是 $x_i$ 本身的倍數,也就是分數
  \[ \frac{x_{i-1}+x_{i+1}}{x_i} = k_i \]
  是一個整數,其中指定 $x_0 = x_n$, $x_{n+1} = x_1$. 試證:所有倍數和 $k_1 + k_2 + \dots + k_n$ 滿足不等式
  \[ 2n \le k_1 + k_2 + \dots + k_n < 3n. \]
  \ii 三角形 $ABC$ 中,  $D$ 與 $E$ 分別為角 $A$ 和角 $B$ 的角平分線與對邊的交點。將一菱形內接與四邊形 $AEDB$ 中,且菱形的定點分別位於 $AEDB$ 不同的邊上。 設 $\phi$ 為此菱形非鈍角的內角。 證明 $\phi \le \max \left\{ \angle BAC, \angle ABC \right\}$. % ISL G3
\end{enumerate}
I'm very dismayed to see that \#2 is a geometric inequality. \#1 looks like combinatorics of the algebraic flavor, so I begin by working on that. I do switch between \#1 and \#2 somewhat but as my work on \#2 was largely useless I'll just focus on my solution to \#1.
I'm told the main idea of \#2 is that for any point $P$ on $EF$, the distance to $BC$ from $P$ equals the sum of those to $AB$ and $AC$. As if anyone would think of this during a quiz\dots

Anyways, let me discuss \#1. I think this problem exemplifies something I'm really starting to appreciate after the Taiwan TST's: \emph{scouting is really damn important}. The problem gives the bounds
\[ 2n \le k_1 + k_2 + \dots + k_n < 3n. \]
The first thing I did was set all the $x_i$ equal to each other. Then $k_i \equiv 2$, and this reached the lower bound of $2n$.
I thus realized that simple AM-GM gave the left-hand side.

The hard part was the right-hand side. At first I thought this was one of those loose inequalities where you have to do some blatant estimates to get something to work with. I immediately had the idea of looking at maximal terms, and I soon found that if the values were not all equal then the largest term was the sum of its two neighbors (i.e.\ $k_i = 1$ if $x_i$ is maximal). But I couldn't see how to finish the problem from there.

I decided to scout anyways -- this turned out to be absolutely crucial. As I looked at cases for $n=4$, I found one, then several, that actually hit $11$. That made me much more nervous. Then I realized that the $(1,2,3,4)$ I had used as one of the examples for $n=4$ generalized readily: for any $n$, the sequence $(1,2,\dots,n)$ achieves $3n-1$. Now I realized the bound was actually sharp, and I was \emph{very} scared.

I looked again at the case $n=8$ of my construction. What happened if I adjusted some terms? Starting from $8$ and $7$, the next value could be $6$, but I decided to try a different value modulo $7$ -- let's say $13$, the next smallest.  I decided to then keep descending as normal, obtaining $(8,7,13,6,5,4,3,2,1)$ -- which obtained equality! Wait, what?

If there were these terrible cases of equality\dots I looked at the $13$ that I had inserted. It suddenly hit me: when I removed the $13$, the sum decreased by exactly $3$. This could only be induction, and I had already made the observation I needed about maximality earlier!

\subsection{TST III, Day 1}
\begin{enumerate}
  \ii 令 $\RR$ 表示實數所稱的集合。 定義集合 $S = \left\{ 1,-1 \right\}$ 與函數 $\sign \colon \RR \to S$ 如下:
  \[ \sign(x) =
    \begin{cases}
      1 & \text{if } x \ge 0; \\
      -1 & \text{if } x < 0.
    \end{cases} \]
  給定奇數 $n$. 試問是否存在 $n^2+n$ 個實數 $a_{ij}, b_i \in S$ ($1 \le i \le j \le n$), 使得任意 $n$ 個數 $x_1, \dots, x_n \in S$, 利用下式
  \begin{align*}
    y_i &= \sign \left( \sum_{j=1}^n a_{ij} x_i \right), \quad \forall 1 \le i \le n; \\
    z &= \sign \left( \sum_{i=1}^n y_i b_i \right)
  \end{align*}
  計算出對應的 $z$ 值恆等於 $x_1x_2 \dots x_n$.
  \ii 試問:是否存在無窮多個整數 $a_1$, $a_2$, $a_3$, \dots 及正整數 $N$, 其中 $0 < a_i < 10$, 使得對於所有正整數 $k > N$,
  \[ \sum_{i=1}^k a_i 10^{i-1} \]
  都是完全平方數? % ISL N4
  \ii 設點 $M$ 為三角形 $ABC$ 的外接圓上一點。 自 $M$ 引對三角形 $ABC$
  的內切圓相切的(兩條)直線,分別角 $BC$ 於 $X_1$, $X_2$.
  證明三角形 $MX_1X_2$ 的外接圓與 $ABC$ 的外接圓的第二個(及不同於 $M$ 的那個交點)就是
  $ABC$ 的與角 $A$ 內的偽內接圓的切點。
\end{enumerate}
\#1 was some convoluted sequences problem. \#2 was a fairly natural number theory, and \#3 was a hard geometry that nobody solved it. You needed a lemma about the point $T$ which I wasn't aware of; after this, an inversion around the incircle makes the problem into something doable (meaning I think I could have gotten it during the test if I had known). Without it, though, I can't see any reasonable way to proceed. I am glad that gave up on the geo after the first half hour after realizing the chance that I solved it was practically nil (given my track record at these camps).

I move on to \#2, the number theory. It looked like a really natural problem, and I could tell that the difference of squares $\left( x_{k+1}-x_k \right)\left( x_{k+1}+x_k \right) = a_{k+1} \cdot 10^k$ was probably really important.
I wasn't sure what I was doing for a while as I tried random things, but eventually I decided to do my proper scouting and try to actually construct a sequence of squares. And I get pretty far: $25,625,5625,15625$ works. I almost convinced myself the answer was actually yes. Almost.

At this point I started looking at powers of $5$ (can you guess why?), and it occurs to me that it's very hard to have $v_5(x_{k+1}^2 - x_k^2)$ odd. This would force the $5$-adic valuations of the two $x_i$ to be equal. But this causes problems with the next line when the $5$-adic valuation isn't big enough. Following throw I'm able to prove that this situation can never happen.

That means the digits have to alternate between $5$ and not $5$ after a while. I try to do more stuff with $5$-adics but nothing happens. I then have the idea of taking modulo $3$, whence I discover the squares have to alternate between zero and nonzero modulo $3$. I keep trying tons of ridiculous stuff over the next hour or so trying to get this case to work, and eventually I do manage to get something, but I think I could have just taken modulo $9$ at this point: for $n$ of the right parity,
\[ 0 \equiv 0-0 \equiv x_{n+2}^2 - x_n^2 = \left( 50 + a_{n+1} \right) \cdot 10^N \pmod{9} \]
is enough to force $a_{n+1} = 4$. (I took modulo $9$ at the very end, after working to show the digits were either $1$ or $4$, to pin it down to $4$\dots oops.) So that means the digits must alternate between $4$ and $5$, but this is an easy contradiction modulo a lot of things, such as $11$.

I'm not sure why it didn't occur to me sooner that mods would be useful for computing the other $a_i$. Oh well.

I spend the rest of the time flailing on \#1. No avail. $1$ problem again. These tests are hard\dots



\subsection{TST III, Day 2}
\begin{enumerate}
  \setcounter{enumi}{3}
  \ii 設三角形 $ABC$ 中有 $\angle B > \angle C$. 設點 $P$ 和 $Q$ 為直線 $AC$ 上相異兩點,滿足 $\angle PBA = \angle QBA = \angle ACB$, 且 $A$ 點位於 $P$ 與 $C$ 點之間。 在線段 $BQ$ 取一點 $D$ 使得 $PD = PB$.
  令射線 $AD$ 與 $\triangle ABC$ 的外接圓交於 $R$ 點 ($R \neq A$). 證明 $QB = QR$. % ISL G4
  \ii 令 $n$ 是以正整數, 考慮以正整數 $a_1$, $a_2$, \dots, $a_n$. 將此數列延伸為有周期的無窮數列,對每個 $i \ge 1$ 都定義 $a_{n+1} = a_i$. 若
  \[ a_1 \le a_2 \le \dots \le a_n \le a_1 + n \]
  且對於 $i = 1,2,\dots,n$ 都有
  \[ a_{a_i} \le n+i-1 \]
  試證
  \[ a_1 + \dots + a_n \le n^2. \]
  % ISL A4
  \ii 甲、乙兩人在實數線上玩以下著色遊戲。甲有一桶顏料四單位,其中 $p$ 單位的顏料剛好可以塗滿以長度為 $p$ 的閉區間。
  每回合,甲先指定一個正整數 $m$, 並給乙 $\frac{1}{2^m}$ 單位的顏料。
  接着,乙選整數 $k$, 並將 $\frac{k}{2^m}$ 到 $\frac{k+1}{2^m}$ 塗滿(此區間可能有一部分在之間的回合中已經被塗過)。
  如果桶子空了但 $[0,1]$ 區間還沒被塗滿,則甲獲勝。 \\
  試問:甲是否有在有限回合內獲勝的必勝法? % ISL C8
\end{enumerate}

The first problem (which apparently is an early shortlist problem) managed to destroy me. It was evident that complex numbers could finish this within an hour: just show that $\ol{OQ} \perp \ol{BR}$, where $O$ is the circumcenter of $ABC$. Now all the points are readily computable.
Somehow, though, I managed to screw up the calculations multiple times over several hours, and ended up not solving it. I still don't know where my computation screwed up\dots I'll find out soon. But indeed, this was quite embarrassing. I guess I've forgotten all of my synthetic geometry.

I didn't really attempt \#6 since it seemed like all the tests had been too hard so far, and I didn't feel (upon reading it) that this was going to be an exception. Indeed, only one person got any credit for this problem at all (a 3). So I was fine there.

On the other hand, I found \#5 to be a very easy problem. After playing around for a while, you find a bunch of places where equality holds; for example when $n=10$, the following sequence achieves equality:
\[ \left( 3,6,8,11,11,11,12,12,13,13 \right). \]
Here we have more or less arbitrarily selected $3$, $6$, $8$ for the beginning; after that we pushed each of the subsequent ``runs'' to the largest permissible value when they didn't matter anymore.
Furthermore, adjusting the numbers a little doesn't change the sum. For example, we can increase the $a_1 = 3$ to a $4$, but this forces $a_4$ to decrease to $10$, which ``locks'' the $13$ at the end in place, and hence the sum is still $100$.

The main finding is that virtually all the little adjustments of the type described above lead to changes which perfectly cancel out, as long as you maximally increase any ``loose'' values; {i.e.} those that, when increased, don't cause any other terms to change. (And there are really few such loose terms).
This kind of forces the sum to evaluate ``exactly'' when we add up the weak inequalities (those that represent pushing the weak terms), and in fact I put down this problem knowing I had solved it before I bothered to write out the solution some time later: if $k=a_1$ then
\begin{align*}
  a_1 + \dots + a_k &= a_1 + \dots + a_k \\
  a_{a_1+1} + \dots + a_{a_2} &\le \left( a_2-a_1 \right)\left( n+1 \right) \\
  a_{a_2+1} + \dots + a_{a_3} &\le (a_3-a_2)(n+2) \\
  &\phantom\le\vdots \\
  a_{a_{k-1}+1} + \dots + a_{a_k} &\le \left( a_k-a_{k-1} \right)(n+k-1) \\
  a_{a_k+1} + \dots + a_n &\le (n-a_k)(n+k)
\end{align*}
Here we've used $a_{a_m+1} + \dots + a_{a_{m+1}} \le a_{a_{m+1}} + \dots + a_{a_{m+1}} \le \left( a_{m+1}-a_m \right)a_{a_{m+1}} \le \left( a_{m+1}-a_m \right)(n+m)$. Summing yields the conclusion.

\end{document}
