\documentclass[11pt]{scrartcl}
\usepackage[sexy]{evan}

\usepackage{answers}
\Newassociation{hint}{hintitem}{all-hints}
\renewcommand{\solutionextension}{out}
\renewenvironment{hintitem}[1]{\item[\bfseries #1.]}{}

\begin{document}
\title{The Incenter/Excenter Lemma}
\date{6 August 2016}
\maketitle

In this short note, we'll be considering the following useful lemma.
\begin{lemma*}
  Let $ABC$ be a triangle with incenter $I$, $A$-excenter $I_A$,
  and denote by $L$ the midpoint of arc $BC$.
  Show that $L$ is the center of a circle through $I$, $I_A$, $B$, $C$.
\end{lemma*}
\begin{center}
  \begin{asy}
    pair A = dir(110);
    pair B = dir(210);
    pair C = dir(330);
    pair I = incenter(A, B, C);
    draw(A--B--C--cycle);
    draw(unitcircle);
    pair L = dir(270);
    pair I_A = 2*L-I;
    draw(CP(L, I), blue);

    draw(A--I_A, dotted);

    dot("$A$", A, dir(A));
    dot("$B$", B, dir(190));
    dot("$C$", C, dir(-10));
    dot("$I$", I, dir(60));
    dot("$L$", L, dir(225));
    dot("$I_A$", I_A, dir(I_A));

    /* Source generated by TSQ */
  \end{asy}
\end{center}
\begin{proof}
  This is just angle chasing.
  Let $A = \angle BAC$, $B = \angle CBA$, $C = \angle ACB$,
  and note that $A$, $I$, $L$ are collinear (as $L$ is on the angle bisector).
  We are going to show that $LB = LI$, the other cases being similar.

  First, notice that
  \[ \angle LBI = \angle LBC + \angle CBI
    = \angle LAC + \angle CBI
    = \angle IAC + \angle CBI
    = \half A + \half B.
    \]
  However,
  \[ \angle BIL = \angle BAI + \angle ABI
    = \half A + \half B.
    \]
  Hence, $\triangle BIL$ is isosceles.
  So $LB = LI$.
  The rest of the proof proceeds along these lines.
\end{proof}

Now, let's see where this lemma has come up before\dots
\eject

\Opensolutionfile{all-hints}

\section{Mild Embarrassments}
\begin{problem}
  [USAMO 1988] Triangle $ABC$ has incenter $I$.
  Consider the triangle whose vertices are the circumcenters of $\triangle IAB$, $\triangle IBC$, $\triangle ICA$.
  Show that its circumcenter coincides with the circumcenter of $\triangle ABC$.
  \begin{hint}
    Tautological.
  \end{hint}
\end{problem}
\begin{problem}
  [CGMO 2012] The incircle of a triangle $ABC$ is tangent to sides $AB$ and $AC$ at $D$ and $E$ respectively, and $O$ is the circumcenter of triangle $BCI$. Prove that $\angle ODB = \angle OEC$.
  \begin{hint}
    Who is $O$?
  \end{hint}
\end{problem}
\begin{problem}
  [CHMMC Spring 2012] In triangle $ABC$, the angle bisector of $\angle A$
  meets the perpendicular bisector of $\overline{BC}$ at point $D$.
  The angle bisector of $\angle B$ meets the perpendicular bisector
  of $\overline{AC}$ at point $E$.
  Let $F$ be the intersection of the perpendicular bisectors of $\overline{BC}$ and $\overline{AC}$.
  Find $DF$, given that $\angle ADF = 5^{\circ}$,
  $\angle BEF = 10^{\circ}$ and $AC = 3$.
  \begin{hint}
    Point $F$ is the circumcenter of $\triangle ABC$.
    Who are $D$ and $E$?
  \end{hint}
\end{problem}
\begin{problem}
  [Nine-Point Circle]
  Let $ABC$ be an acute triangle with orthocenter $H$.
  Let $D$, $E$, $F$ be the feet of the altitudes from $A$, $B$, $C$ to the opposite sides.
  Show that the midpoint of $\overline{AH}$ lies on the circumcircle of $\triangle DEF$.
  \begin{hint}
    What is the incenter of $\triangle DEF$?
    What is the $D$-excenter?
  \end{hint}
\end{problem}

\section{Some Short-Answer Problems}
\begin{problem}
  [HMMT 2011] Let $ABCD$ be a cyclic quadrilateral, and suppose that $BC = CD = 2$.
  Let $I$ be the incenter of triangle $ABD$.
  If $AI = 2$ as well, find the minimum value of the length of diagonal $BD$.
  \begin{hint}
    Show that $AC = 4$.
  \end{hint}
\end{problem}
\begin{problem}
  [HMMT 2013] Let triangle $ABC$ satisfy $2BC = AB+AC$ and have incenter $I$ and circumcircle $\omega$.
  Let $D$ be the intersection of $AI$ and $\omega$ (with $A$, $D$ distinct).
  Prove that $I$ is the midpoint of $AD$.
  \begin{hint}
    Apply Ptolemy's Theorem.
  \end{hint}
\end{problem}

\begin{problem}
  [Online Math Open 2014/F19] In triangle $ABC$, $AB=3$, $AC=5$, and $BC=7$.
  Let $E$ be the reflection of $A$ over $\overline{BC}$, and let line $BE$ meet the circumcircle of $ABC$ again at $D$. Let $I$ be the incenter of $\triangle ABD$.
  Compute $\cos \angle AEI$.
  \begin{hint}
    Who is $C$? Erase $E$.
  \end{hint}
\end{problem}

\begin{problem}
  [NIMO 2012] Let $ABXC$ be a cyclic quadrilateral such that $\angle XAB = \angle XAC$.
  Let $I$ be the incenter of triangle $ABC$
  and by $D$ the foot of $I$ on $\overline{BC}$.
  Given $AI=25$, $ID = 7$, and $BC = 14$, find $XI$.
  \begin{hint}
    Apply Ptolemy's Theorem.
  \end{hint}
\end{problem}

\section{Intermediate Examples}
\begin{problem}
  Let $ABC$ be an acute triangle such that $\angle A = 60\dg$.
  Prove that $IH = IO$, where $I$, $H$, $O$ are the incenter, orthocenter, and circumcenter.
  \begin{hint}
    Since $\angle BHC = \angle BIC = \angle BOC = 120\dg$, points $H$ and $O$ now lie on the magic circle too.
    So $IH = IO$ is just an equality of certain arcs.
  \end{hint}
\end{problem}

\begin{problem}
  [IMO 2006] Let $ABC$ be a triangle with incenter $I$. A point $P$ in the interior of the triangle satisfies \[\angle PBA+\angle PCA = \angle PBC+\angle PCB.\] Show that $AP \geq AI$, and that equality holds if and only if $P=I$.
  \begin{hint}
    Use the angle condition to show that $P$ also lies on the magic circle.
  \end{hint}
\end{problem}

%\begin{problem}
%  [Online Math Open 2012/F24] In scalene $\triangle ABC$, $I$ is the incenter,
%  $I_a$ is the $A$-excenter, $D$ is the midpoint of arc $BC$ of the circumcircle of $ABC$
%  not containing $A$, and $M$ is the midpoint of side $BC$.
%  Extend ray $IM$ past $M$ to point $P$ such that $IM = MP$.
%  Let $Q$ be the intersection of $DP$ and $MI_a$,
%  and $R$ be the point on the line $MI_a$ such that $AR\parallel DP$.
%  Given that $\frac{AI_a}{AI}=9$, the ratio $\frac{QM} {RI_a}$ can be expressed
%  in the form $\frac{m}{n}$ for two relatively prime positive integers $m$, $n$.
%  Compute $m+n$.
%\end{problem}

\begin{problem}
  [APMO 2007] In triangle $ABC$, we have $AB > AC$ and $\angle A = 60\dg$.
  Let $I$ and $H$ denote the incenter and orthocenter of the triangle.
  Show that $2\angle AHI = 3\angle B$.
  \begin{hint}
    The point $H$ lies on the magic circle.
    So $\angle IHC = 180\dg - \angle IBC$.
  \end{hint}
\end{problem}

\begin{problem}
  [ELMO 2013, Evan Chen]
  Triangle $ABC$ is inscribed in circle $\omega$.
  A circle with chord $BC$ intersects segments $AB$ and $AC$ again at $S$ and $R$, respectively.
  Segments $BR$ and $CS$ meet at $L$, and rays $LR$ and $LS$ intersect $\omega$ at $D$ and $E$, respectively.
  The internal angle bisector of $\angle BDE$ meets line $ER$ at $K$.
  Prove that if $BE = BR$, then $\angle ELK = \tfrac{1}{2} \angle BCD$.
  \begin{hint}
    You need to do quite a bit of angle chasing.
    Show that $R$ is the incenter of $\triangle CDE$.
    Who is $B$?
  \end{hint}
\end{problem}

\begin{problem}
  [Online Math Open 2012/F27]
  Let $ABC$ be a triangle with circumcircle $\omega$.
  Let the bisector of $\angle ABC$ meet segment $AC$ at $D$ and circle $\omega$ at $M\neq B$.
  The circumcircle of $\triangle BDC$ meets line $AB$ at $E \neq B$,
  and $CE$ meets $\omega$ at $P\neq C$.
  The bisector of $\angle PMC$ meets segment $AC$ at $Q \neq C$.
  Given that $PQ = MC$, determine the degree measure of $\angle ABC$.
  \begin{hint}
    Both $M$ and $P$ are arc midpoints. (Why?)
  \end{hint}
\end{problem}


\section{Harder Tasks}
\begin{problem}
  [Iran 2001] Let $ABC$ be a triangle with incenter $I$ and $A$-excenter $I_A$.
  Let $M$ be the midpoint of arc $BC$ not containing $A$, and let $N$ denote the midpoint of arc $MBA$.
  Lines $NI$ and $NI_A$ intersect the circumcircle of $ABC$ at $S$ and $T$.
  Prove that the lines $ST$, $BC$ and $AI$ are concurrent.
  \begin{hint}
    First show that $S$, $T$, $I$, $I_A$ are concyclic, say by $NI \cdot NS = NM^2 = NI_A \cdot NT$.
  \end{hint}
\end{problem}

\begin{problem}
  [Online Math Open 2014/F26] Let $ABC$ be a triangle with $AB=26$, $AC=28$, $BC=30$.
  Let $X$, $Y$, $Z$ be the midpoints of arcs $BC$, $CA$, $AB$ (not containing the opposite vertices)
  respectively on the circumcircle of $ABC$.
  Let $P$ be the midpoint of arc $BC$ containing point $A$.
  Suppose lines $BP$ and $XZ$ meet at $M$ , while lines $CP$ and $XY$ meet at $N$.
  Find the square of the distance from $X$ to $MN$.
  \begin{hint}
    Add the incenter $I$.
    Line $MN$ is a tangent.
  \end{hint}
\end{problem}

\begin{problem}
  [Euler]
  Let $ABC$ be a triangle with incenter $I$ and circumcenter $O$.
  Show that $IO^2 = R(R-2r)$, where $R$ and $r$ are the circumradius and inradius of $\triangle ABC$, respectively.
  \begin{hint}
    Add in point $L$, the midpoint of arc $BC$.
    By Power of a Point, it's equivalent to prove $AI \cdot IL = 2Rr$,
    which can be done with similar triangles.
  \end{hint}
\end{problem}

\begin{problem}
  [IMO 2010]
  Let $I$ be the incenter of a triangle $ABC$ and let $\Gamma$ be its circumcircle.
  Let the line $AI$ intersect $\Gamma$ again at $D$.
  Let $E$ be a point on the arc $BDC$ and $F$ a point on the side $BC$ such that
  \[ \angle BAF = \angle CAE < \tfrac12 \angle BAC . \]
  Finally, let $G$ be the midpoint of $\ol{IF}$. Prove that $\ol{DG}$ and $\ol{EI}$ intersect on $\Gamma$.
  \begin{hint}
    Take a homothety with ratio $2$ at $I$.
    This sends $G$ to $F$ and $D$ to the $A$-excenter.
  \end{hint}
\end{problem}

\section{Bonus Problems}
\begin{problem}
  [Russia 2014] Let $ABC$ be a triangle with $AB>BC$ and circumcircle $\Omega$.
  Points $M$, $N$ lie on the sides $AB$, $BC$ respectively, such that $AM=CN$.
  Lines $MN$ and $AC$ meet at $K$.
  Let $P$ be the incenter of the triangle $AMK$, and let $Q$ be the $K$-excenter of the triangle $CNK$.
  If $R$ is midpoint of arc $ABC$ of $\Omega$ then prove that $RP=RQ$.
  \begin{hint}
    Construct arc midpoints on the circumcircles of both $\triangle AMJ$ and $\triangle CNK$.
    Use spiral similarity at $R$.
  \end{hint}
\end{problem}
\begin{problem}
  Let $ABC$ be a triangle with circumcircle $\Omega$, and let $D$ be any point on $\ol{BC}$.
  We draw a \emph{curvilinear incircle} tangent to $\overline{AD}$ at $L$, to $\overline{BC}$ at $K$
  and internally tangent to $\Omega$.
  Show that the incenter of triangle $ABC$ lies on $\overline{KL}$.
  \begin{hint}
    Let the tangency point to $\Omega$ be $T$, let $M$ be the midpoint of arc $BC$,
    and let lines $KL$ and $AM$ meet at $I$.
    Show that $M$, $K$, $T$ are collinear.
    Show that $ALIT$ is cyclic.
    Prove that $MI^2 = MK \cdot MT = MC^2 = MI^2$.
  \end{hint}
\end{problem}


\eject

\section{Hints to the Problems}
\Closesolutionfile{all-hints}
\begin{enumerate}
\input{all-hints.out}
\end{enumerate}

\end{document}
