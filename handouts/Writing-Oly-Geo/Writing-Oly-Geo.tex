\documentclass[11pt]{scrartcl}
\usepackage[mdthm]{evan}

\declaretheorem[style=thmrecbox,name=Problem]{olyprob}

\begin{document}
\title{Writing Olympiad Geometry Problems}
\date{22 December 2015}
\maketitle

\begin{quote}
  \sffamily\small
  You can use a wide range of wild, cultivated or supermarket greens in this recipe.
  Consider nettles, beet tops, turnip tops, spinach, or watercress in place of chard.
  The combination is also up to you so choose the ones you like most.

  \medskip

  --- Y. Ottolenghi. Plenty More
\end{quote}

%%fakesection Introduction
This is a PDF of my blog post \url{http://wp.me/p3jiSp-8r}.
Here I'll describe how I come up with geometry proposals
for olympiad-style contests. In particular, I'll go into detail
about how I created the following two problems,
which were the first olympiad problems which I got onto a contest.
Note that I don't claim this is the only way to write such problems,
it just happens to be the approach I use, and has consistently
gotten me reasonably good results.

\begin{olyprob}[USA December TST for 56th IMO]
  Let $ABC$ be a triangle with incenter $I$ whose incircle is tangent to
  $\overline{BC}$, $\overline{CA}$, $\overline{AB}$ at $D$, $E$, $F$,
  respectively.  Denote by $M$ the midpoint of $\overline{BC}$ and
  let $P$ be a point in the interior of $\triangle ABC$
  so that $MD = MP$ and $\angle PAB = \angle PAC$.
  Let $Q$ be a point on the incircle such that $\angle AQD = 90^{\circ}$.
  Prove that either $\angle PQE = 90^{\circ}$ or $\angle PQF = 90^{\circ}$.
\end{olyprob}


\begin{olyprob}
  [Taiwan TST Quiz for 56th IMO]
  In scalene triangle $ABC$ with incenter $I$, the incircle is tangent to
  sides $CA$ and $AB$ at points $E$ and $F$.
  The tangents to the circumcircle of $\triangle AEF$ at $E$ and $F$ meet at $S$.
  Lines $EF$ and $BC$ intersect at $T$.
  Prove that the circle with diameter $ST$ is orthogonal to
  the nine-point circle of $\triangle BIC$.
\end{olyprob}

\begin{center}
  \includegraphics[scale=0.8]{DecTST/statement.pdf}
  \includegraphics{TWTST/statement.pdf}
\end{center}


\section{General Procedure}
Here are the main ingredients you'll need.
\begin{itemize}
  \ii The ability to consistently solve medium to hard
  olympiad geometry problems. The intuition you have from being
  a contestant proves valuable when you go about looking for things.

  \ii In particular, a good eye:
  in an accurate diagram, you should be able to notice
  if three points look collinear or if four points are concyclic,
  and so on. Fortunately, this is something you'll hopefully have
  just from having done enough olympiad problems.

  \ii Geogebra, or some other software that will let you quickly
  draw and edit diagrams.
\end{itemize}

With that in mind, here's the gist of what you do.
\begin{enumerate}
  \ii Start with a configuration of your choice;
  something that has a bit of nontrivial structure in it,
  and add something more to it.
  For example, you might draw a triangle with its incircle
  and then add in the excircle tangency point,
  and the circle centered at $BC$ passing through both points
  (taking advantage of the fact that the two tangency points
  are equidistant from $B$ and $C$).

  \ii Start playing around, adding in points and so on
  to see if anything interesting happens.
  You might be guided by some actual geometry constructions:
  for example, if you know that the starting configuration
  has a harmonic bundle in it,
  you might project this bundle to obtain the new points to play with.

  \ii Keep going with this until you find something unexpected:
  three points are collinear, four points are cyclic, or so on.
  Perturb the diagram to make sure your conjecture looks like
  it's true in all cases.

  \ii Figure out why this coincidence happened.
  This will probably add more points to you figure,
  since you often need to construct more auxiliary points
  to prove the conjecture that you have found.

  \ii Repeat the previous two steps to your satisfaction.

  \ii Once you are happy with what you have,
  you have a nontrivial statement and probably
  several things that are equivalent to it.
  Pick the one that is most elegant (or hardest), and erase
  auxiliary points you added that are not needed for the
  problem statement.

  \ii Look for other ways to reduce the number of points
  even further, by finding other equivalent formulations
  that have fewer points.
\end{enumerate}
Or shorter yet: build up, then tear down.

None of this makes sense written this abstractly,
so now let me walk you through the two problems I wrote.

\eject

\section{The December TST Problem}
In this narrative, the point names might be a little strange at first,
because (to make the story follow-able) I used the point
names that ended up in the final problem,
rather than ones I initially gave.
Please bear with me!

I began by drawing a triangle $ABC$ (always a good start\dots)
and its incircle, tangent to side $BC$ at $D$.
Then, I added in the excircle touch point $T$,
and drew in the circle with diameter $DT$, which was
centered at the midpoint $M$.
This was a coy way of using the fact that $MD = MT$;
I wanted to see whether it would give me anything interesting.

So, I now had the following picture.
\begin{center}
  \includegraphics{DecTST/initial.pdf}
\end{center}

Now I had two circles intersecting at a single point $D$,
so I added in $Q$, the second intersection.
But really, this point $Q$ can be thought of another way.
If we let $DS$ be the diameter of the incircle,
then as $DT$ is the other diameter, $Q$ is actually just the
foot of the altitude from $D$ to line $ST$.

But recall that $A$, $S$, $T$ are collinear!
(Again, this is why it's helpful to be familiar with ``standard''
contest configurations; you see these kind of things immediately.)
So $Q$ in fact lies on line $AT$.

\begin{center}
  \includegraphics{DecTST/point-q.pdf}
\end{center}

This was pretty cool, though not yet interesting enough to be
a contest problem. So I looked for most things that might be true.

I don't remember what I tried next; it didn't do anything interesting.
But I do remember the thing I tried after that:
I drew in the angle bisector, line $AI$.
And then, I noticed a big coincidence:
the first intersection of $AI$ with the circle with diameter $DT$
seemed to lie on line $DE$!
I was initially confused by this; it didn't seem like it could
possibly be true due to symmetry reasons.
But in my diagram, it was indeed correct.
A moment later, I realized the reason why this was plausible:
in fact, the second intersection of line $AI$ with the circle
was on line $DF$.

\begin{center}
  \includegraphics{DecTST/collin.pdf}
\end{center}

Now, I could not see quickly at all why this was true.
So I started trying to prove it, but initially failed:
however, I managed to show (via angle chasing) that
\[ D, P, E \text{ collinear}
  \iff \angle PQE = 90\dg. \]
So, at least I had an interesting equivalent statement.
\begin{center}
  \includegraphics{DecTST/almost-sol.pdf}
\end{center}

After another half hour of trying to prove my conjecture,
I finally realized what was happening.
The point $P$ was the one attached to a particular lemma:
the $A$-bisector, $B$-midline, and $C$ touch-chord are concurrent,
and from this $MD = MP$ just follows by some similar triangles.
So, drawing in the point $N$ (the midpoint of $AB$),
I had the full configuration which gave the answer to my conjecture.

\begin{center}
  \includegraphics{DecTST/full-sol.pdf}
\end{center}

Finally, I had to clean up the mess that I had made.
How could I do this?
Well, the points $N$, $S$ could be eliminated easily enough.
And we could re-define $Q$ to be a point on the incircle
such that $\angle AQD = 90\dg$.
This actually eliminated the green circle and point $T$ altogether,
provided we defined $P$ by just saying that it was on the angle
bisector, and that $MD = MP$.
(So while the circle was still implicit in the condition $MD = MP$,
it was no longer explicitly part of the problem.)

Finally, we could even remove the line through $D$, $P$ and $E$;
we ask the contestant to prove $\angle PQE = 90\dg$.

\begin{center}
  \includegraphics{DecTST/statement.pdf}
\end{center}

And that was it!

\eject

\section{The Taiwan TST Problem}
In fact, the starting point of this problem was the same lemma
which provided the key to the previous solution:
the circle with diameter $BC$ intersects the $B$ and $C$ bisectors
on the $A$ touch chord.
Thus, we had the following diagram.
\begin{center}
  \includegraphics{TWTST/iran.pdf}
\end{center}
The main idea I had was to look at the points $D$, $X$, $Y$
in conjunction with each other.
Specifically, this was the orthic triangle of $\triangle BIC$,
a situation which I had remembered from working on
Iran TST 2009, Problem 9.
So, I decided to see what would happen
if I drew in the nine-point circle of $\triangle BIC$.
Naturally, this induces the midpoint $M$ of $BC$.
\begin{center}
  \includegraphics{TWTST/nine-point.pdf}
\end{center}
At this point, notice (or recall!) that line $AM$
is concurrent with lines $DI$ and $EF$.
\begin{center}
  \includegraphics{TWTST/median.pdf}
\end{center}
So the nine-point circle of the problem is very tied
down to the triangle $BIC$.
Now, since I was in the mood for something projective,
I constructed the point $T$,
the intersection of lines $EF$ and $BC$.
In fact, what I was trying to do was take perspectivity through $I$.
From this we actually deduce that $(T,K;X,Y)$ is a harmonic bundle.
\begin{center}
  \includegraphics{TWTST/add-T.pdf}
\end{center}

Now, what could I do with this picture?
I played around looking for some coincidences,
but none immediately presented themselves.
But I was enticed by the point $T$, which was somehow
related to the cyclic complete quadrilateral $XYMD$.
So, I went ahead and constructed the pole of $T$
to the nine-point circle, letting it hit line $BC$ at $L$.
This was aimed at ``completing'' the picture
of a cyclic quadrilateral and the pole of an intersection of two sides.
In particular, $(T,L;D,M)$ was harmonic too.
\begin{center}
  \includegraphics{TWTST/almost.pdf}
\end{center}
I spent a long time thinking about how I could make this into a problem.
I unfortunately don't remember exactly what things I tried,
other than the fact that I was taking a lot of perspectivity.
In particular, the ``busiest'' point in the picture is $K$,
so it makes sense to try and take perspectives through it.
Especially enticing was the harmonic bundle
\[
  \left( \overline{KT}, \overline{KL}; \overline{KD}, \overline{KM} \right)
  = -1.
\]
How could I use this to get a nice result?

Finally about half an hour I got the right idea.
We could take this bundle and intersect it with the ray $AI$!
Now, letting $N$ be the midpoint $EF$,
we find that three of the points in the harmonic bundle we obtain
are $A$, $I$, and $N$; let $S$ be the fourth point,
which is the intersection of line $KL$ with $AI$.
Then by hypothesis, we ought to have $(A,I;N,S) = -1$.
But from this we know exactly what the point $S$.
Just look at the circumcircle of triangle $AEF$:
as this has diameter $AI$, we see that $S$
is the intersection of the tangents at $E$ and $F$.
\begin{center}
  \includegraphics{TWTST/final-sol.pdf}
\end{center}
Consequently, we know that the point $S$,
defined very naturally in terms of the original picture,
lies on the polar of $T$ to the nine-point circle.
By simply asking the contestant to prove this,
we thus eliminate all the points $K$, $M$, $D$, $N$, $I$, $X$, and $Y$
completely from the picture, leaving only the nine-point circle.
Finally, instead of directly asking the contestant to show
that $T$ lies on the polar of $S$, one can rephrase the problem as saying
``the circle with diameter $ST$ is orthogonal to the
nine-point circle of $\triangle BIC$'', concealing all the work
that went into the creation of the problem.
\begin{center}
  \includegraphics{TWTST/statement.pdf}
\end{center}
Fantastic.

\end{document}
