\documentclass{scrartcl}
\usepackage[sexy]{evan}

\begin{asydef}
// Draws triangle for Chinese dumbass notation.
// real[] M are the entries of the triangle, from left to right, up to down.
// deg is the degree (one less than the number of rows).

void dumbass(real[] M, int deg, pair base=(0,0), real dist = 1) {
  int n = deg + 1;
  int k = 0;
  for (int i=0; i<n; i += 1) {
    for (int j=0; j<=i; j += 1) {
      pair P = ( 2*dist*j + dist*(n-i), (n-i)*dist) + base;
      // string coeff = "$" + string(M[k]) + "$";
      string coeff = string(M[k]);
      label(coeff, P);
      k += 1;
    }
  }
  return;
}
\end{asydef}

\begin{document}
\title{Supersums of Square-Weights (SOS)}
\subtitle{A Dumbass's Perspective}
\date{6 January 2013}
\maketitle

\begin{abstract}
  The intended audience should already be familiar with Muirhead and Schur.
  This article is centered around two powerful tools
  for solving polynomial-type inequalities:
  the SOS technique and Chinese Dumbass Notation.
\end{abstract}

\section{Introducing SOS}
\begin{quote}
  \emph{The whole point of SOS is so that you can use AM-GM backwards}. \\
  --- MOP 2012
\end{quote}

\subsection{A Motivating Example}
\begin{example}
  For $a,b,c \ge 0$, prove that \[ \frac{2}{3}(a^2+b^2+c^2)^2 \ge a^3(b+c) + b^3(c+a) + c^3(a+b). \]
\end{example}
\begin{soln}
  Schur/Muirhead is not sufficient here.
  Thinking wishfully, we can write down the two inequalities
  \begin{align*}
    \half \left( a^4+b^4 \right) &\ge \half (a^3b+ab^3) \\
    2a^2b^2 &\le a^3b + ab^3
  \end{align*}

  The cyclic sum of these two inequalities will yield the desired quantity.
  Unfortunately, we're not allowed to add two different inequalities in different directions.
  Or can we?

  Rewrite the two inequalities as:
  \begin{align*}
    \half \left( a^2+ab+b^2 \right)(a-b)^2 &\ge 0 \\
    (-ab)(a-b)^2 &\le 0 \\
  \end{align*}

  Now their ``sum'' is
  \[ \half (a^2-ab+b^2)(a-b)^2 \ge 0 \]
  which is true!
  \emph{So our attempt to add two inequalities pointing in different directions
  has led to the identity}:
  \[ \half \cycsum (a^2-ab+b^2)(a-b)^2 \ge 0 \]
  and the problem is solved.
\end{soln}

This is a strong way to motivate the method:
by factoring out a square $(a-b)^2$, we can ``measure'' the strength of an inequality
that we are trying to apply, and thus we may add and subtract these square weights
without regards to direction.

\subsection{The SOS Theorem}
The basic idea is to write an inequality in the form
\[ S_a (b-c)^2 + S_b (c-a)^2 + S_c (a-b)^2 \ge 0. \]
If it so happens that $S_a$, $S_b$ and $S_c$ are nonnegative, then we are done.
For sharper inequalities this may not be the case;
however, most of the time we can still finish using some minor adjustments.
Some of the nicer estimates are provided below.
They are drawn from \cite{sos}.

\begin{theorem}[Pham Kim Hung]
  \label{thm:sos}
  Consider the sum $S = S_a(b-c)^2 + S_b(c-a)^2 + S_c(a-b)^2$.
  Then $S \ge 0$ if any of the following take place:
  \begin{enumerate}[(i)]
    \ii $S_a$, $S_c$, $S_a + 2S_b$, $S_c + 2S_b$ are all nonnegative.
    \ii $S_b$, $S_a+S_b$ and $S_b+S_c$ are all nonnegative when $b$ is the median of $\{a,b,c\}$.
    \ii $S_b$, $S_c$ and $b^2S_a + a^2S_b$ are all nonnegative when $a \ge b \ge c$.
  \end{enumerate}
\end{theorem}
\begin{proof}
  For the first part, the AM-GM $(a+c)^2 + (2b)^2 \ge 4b(a+c)$
  yields $(a-c)^2 \le 2(a-b)^2 + 2(b-c)^2$.
  If $S_b \ge 0$ we are already done; else make the estimate
  \begin{align*}
    S &\ge S_c(a-b)^2 + S_a(c-b)^2 + \left(2(a-b)^2 + 2(b-c)^2\right) S_b \\
    &= (S_c+2S_b)(a-b)^2 + (S_a+2S_b)(c-b)^2 \\ &\ge 0. \\
    \intertext{For the second part,
      we have $(b-a)(b-c) \le 0 \implies (a-c)^2 \ge (a-b)^2 + (b-c)^2$,
      so we make the estimate}
    S &\ge S_a(b-c)^2 + S_c(a-c)^2 + S_b\left( (b-c)^2 + (a-c)^2 \right) \\
    &= (S_a + S_b)(b-c)^2 + (S_b + S_c)(a-c)^2 \\ &\ge 0. \\
    \intertext{The third part follows from observing that when $a \ge b \ge c$ we have
      $\frac{a-c}{b-c} \ge \frac{a}{b}$, so that $(a-c)^2 \ge \frac{a^2}{b^2}(b-c)^2$.
      Hence, we drop the $S_c(a-b)^2$ term to obtain}
    S &\ge S_a(b-c)^2 + S_b(a-c)^2 \\
    &\ge S_a (b-c)^2 + S_b \left( \frac{a^2(b-c)^2}{b^2} \right) \\
    &= \left( \frac{b-c}{b} \right)^2 \left( b^2S_a + a^2S_b \right) \\
    &\ge 0. \qedhere
  \end{align*}
\end{proof}

\section{Chinese Dumbass Notation}
Before proceeding further I would like to make sure the reader is acquainted
with the extremely useful Chinese Dumbass Notation, as described in \cite{cdn}.
Chinese Dumbass Notation is a convenient way of presenting a polynomial inequality,
making it significantly easier to find the suitable applications.

The idea is that the coefficients of an inequality of the form $P(a,b,c) \ge 0$
are held in a triangle of side length $\deg P + 1$,
where the coefficient of the $a^x b^y c^z$ term are located at the point with
\href{https://web.evanchen.cc/handouts/bary/bary-full.pdf}{barycentric coordinates} $(x:y:z)$.
Explicitly for, say, the third degree:
\[
  \begin{array}[t]{ccccccc}
    &&& [a^3] &&& \\
    && [a^2b] && [a^2c] && \\
    & [ab^2] && [abc] && [ac^2] & \\ {}
    [b^3] && [b^2c] && [bc^2] && [c^3]
  \end{array}
\]
To further illustrate the point, here are some common inequalities:
\begin{figure}[h]
  \centering
  \begin{asy}
  real sep = 10;
  size(15cm);
  real[] M = {
    0,
    0,0,
    0,0,0,
    1,-1,-1,1
  };
  dumbass(M,3,(0,7));
  // b^3+c^3 \ge bc(b+c)

  real[] M = {
    1,
    -1,-1,
    1,-1,1
  };
  dumbass(M, 2, (10,7));
  // a^2+b^2+c^2 \ge ab+bc+ca

  real[] M = {
    1,
    -1, -1,
    -1, 3, -1,
    1, -1, -1, 1
  };
  dumbass(M,3, (20,7));


  real[] M = {
    1,
    -2,0,
    1,0,0
  };
  dumbass(M,2, (0,0));
  // (a-b)^2 \ge 0

  real[] M = {
    2,
    -3, -3,
    4, 0, 4,
    -3, 0, 0, -3,
    2,-3,4,-3,2
  };
  dumbass(M,4, (8,0));
  // The first example

  real[] M = {
    1,
    -1, -1,
    0, 1, 0,
    0, 0, 0, 0,
    -1, 1, 0, 1,-1,
    1,-1, 0, 0, -1,1
  };
  dumbass(M,5, (18,0));
  // Degree 5 Schur

  \end{asy}
  \caption{First row: $b^3+c^3 \ge bc(b+c)$, $a^2+b^2+c^2 \ge ab+bc+ca$ and 3rd degree Schur.
  Second row: $(a-b)^2 \ge 0$, our first example, and 5th degree Schur.}
  \label{fig:cdnexamples}
\end{figure}

What makes this notation powerful?
Once one recognizes Schur and AM-GM, then it is visually easy to see
how strong\footnote{In particular, if the last row sums to zero
  then equality holds at $b=c$, $a=0$.} the inequality is.
Furthermore, if one is using SOS, then they can pull out things like $\{1, -2, 1\}$
and $\{1, -1, -1, 1\}$ from rows (things like the upper-left diagram).
One may also recognize $\{1,-4,6,-4,1\}$ as $(a-b)^4$.

Finally, one can often use the identity
$ a^3+b^3+c^3-3abc = \half \cycsum (a+b+c)(b-c)^2 $
to convert a $3$-variable application of AM-GM to SOS form.
But it is usually sufficient to just add/subtract things which are parallel to the sides of the triangle.

\subsection{Chinese Dumbass Notation: An Example}
To impress the usefulness of this notation, we provide an instructive example.
The unfamiliar reader is strongly encouraged to try and solve this problem
(via expansion) before reading the solution below.
\begin{example}[MOP 2011]
  Let $a,b,c$ be positive real numbers such that $a+b+c=3$. Show that
  \[ \frac{1}{a} + \frac{1}{b} + \frac{1}{c} \ge 1 + 2\sqrt{ \frac{a^2+b^2+c^2}{3abc} }. \]
\end{example}

\begin{soln}
  Homogenizing and rearranging, this becomes
  \[ \left( \left( a+b+c \right) \left( \frac{1}{a} + \frac{1}{b} + \frac{1}{c} \right) - 3 \right)^2
    \ge 4\left( \frac{(a+b+c)(a^2+b^2+c^2)}{abc} \right) \]
  Expanding and clearing denominators, this reduces to
  \[ \symsum a^4b^2 + 2\cycsum a^3b^3+ 6a^2b^2c^2 \ge 2 \cycsum a^4bc + 2 \symsum a^3b^2c. \]
  Trying to fight with this mess is no easy task.
  But watch how easy the problem once we put in a triangle:

  \begin{figure}[h]
    \centering
    \begin{asy}
    size(6cm);
    real[] M = {
      0,
      0, 0,
      1, -2, 1,
      2, -2, -2, 2,
      1, -2, 6, -2, 1,
      0, -2, -2, -2, -2, 0,
      0, 0, 1, 2, 1, 0, 0
    };
    dumbass(M, 6);
    \end{asy}
    \caption{K4.1 expanded}
    \label{fig:blackcdn}
  \end{figure}

  It's hard to see without the triangle, but now we notice that third-degree Schur appears upside-down!
  The $\{1, -2, 1\}$ on each side also die off: so this problem is merely the identity
  \[ \cycsum a^4(b-c)^2 + 2\cycsum ab(ab-bc)(ab-ac) \ge 0. \]
  which is evident.
\end{soln}


\section{Some Examples}
\subsection{SOS vs Schur}
\begin{example}
  Let $a,b,c$ be nonnegative reals.
  Show that $a^3+b^3+c^3+3abc \ge a^2(b+c) + b^2(c+a) + c^2(a+b)$.
\end{example}
\begin{soln}
  Double both sides and write the inequality in the triangle:

  \begin{figure}[h]
    \centering
    \begin{asy}
    size(4cm);
    real[] M = {2,
      -2, -2,
      -2, 6, -2,
      2, -2, -2, 2
    };
    dumbass(M, 3);
    \end{asy}
    \caption{Third-degree Schur}
    \label{fig:deg3schur}
  \end{figure}

  We can take out $\cycsum a^3+b^3-a^2b-ab^2 = \cycsum (a+b)(a-b)^2$
  which will leave us a hexagon in the center; which we break up $-\cycsum a(b-c)^2$.
  Hence the original problem is equivalent to
  \[ \half \cycsum (b+c-a) (b-c)^2 \ge 0. \]
  If $a \ge b \ge c$ then it is trivial to see $S_b$, $S_b + S_a$ and $S_b + S_c$ are all nonnegative;
  hence we are done.
\end{soln}
\begin{exercise*}
  Put fourth-degree Schur in SOS form.
\end{exercise*}

In fact, upon discovering $(b-c)(b-a) = \half \left( (b-c)^2 + (b-a)^2 - (a-c)^2 \right)$
for arbitrary $r$ the inequality may be written as
\[ \half \cycsum b^r \left( (b-c)^2 + (b-a)^2 - (a-c)^2 \right) = \half \cycsum (b^r+c^r-a^r)(b-c)^2. \]
which is obviously nonnegative by the SOS criteria.

\subsection{An Asymmetric Inequality}
\begin{example}[Ukraine 2006]
  Let $a,b,c$ be positive reals.
  Prove that \[ 3(a^3+b^3+c^3+abc) \ge 4(a^2b+b^2c+c^2a). \]
\end{example}
\begin{soln}
  Seeking symmetry, we multiply both sides by six and write it as the sum
  in the following figure.
  \begin{figure}[h]
    \centering
    \begin{asy}
    size(8cm);
    real[] M = {
      8,
      -16,8,
      8,0,-16,
      8,-16,8,8
    };
    dumbass(M,3);
    real[] M = {
      10,
      -8,-8,
      -8,18,-8,
      10,-8,-8,10
    };
    dumbass(M,3, (10,0));
    \end{asy}
    \caption{Trying to get symmetry}
    \label{fig:ukraine}
  \end{figure}

  The left triangle is contrived using a multiple of the form $\cycsum a(a-b)^2$
  to get the second addend to be symmetric.
  Now, we can express it as $5 \cycsum (b+c)(b-c)^2 - 3 \cycsum a(b-c)^2$.
  All in all, the total is
  \[ \cycsum (13b+5c-3a)(b-c)^2 \ge 0 \]
  which is not hard.

  To be fair, the two triangles above are already nonnegative as is,
  meaning that simple AM-GM and Schur was sufficient.
\end{soln}

\section{Too Lazy to Expand}
Up to now, we have used SOS on an inequality
only if it could be written in Chinese Dumbass notation.
In fact, with a bit more precision, one can use SOS on inequalities with fractions
\textit{without} expanding.
An extremely useful fact is as follows:
\begin{fact*}
  If a polynomial expression $f(a,b,c)$ is
  \begin{enumerate}[(i)]
    \ii symmetric in $a$ and $b$ (that is, $f(a,b,c) = f(b,a,c)$), and
    \ii zero when $a=b$ (that is, $f(a,a,c) = 0$)
  \end{enumerate}
  then it is divisible by $(a-b)^2$.
\end{fact*}
This means that, for example, the expression
$\frac{a^2}{b+c} + \frac{b^2}{c+a} - \frac{4ab}{a+b+2c}$ is divisible by $(a-b)^2$:
it's just a matter of computation to extract the factor.

\subsection{A Perfect Example}
\begin{example}[Calvin Deng, ELMO Shortlist 2010]
  Let $a,b,c$ be positive reals. Prove that
  \[ \frac{(a-b)(a-c)}{2a^2 + (b+c)^2} + \frac{(b-c)(b-a)}{2b^2 + (c+a)^2} + \frac{(c-a)(c-b)}{2c^2 + (a+b)^2} \geq 0. \]
\end{example}
\begin{soln}
  We could not have asked for more here: the first two fractions are symmetric in $a$ and $b$,
  but even better, it's downright \textit{obvious} that it becomes zero at $a=b$.
  The rest is just courage:
  \begin{align*}
    \frac{(a-b)(a-c)}{2a^2+(b+c)^2} + \frac{(b-c)(b-a)}{2b^2+(c+a)^2}
    &= (a-b) \left( \frac{a-c}{2a^2+(b+c)^2} - \frac{b-c}{2b^2 +(c+a)^2} \right) \\
    &= (a-b) \cdot \frac{(a-c)(2b^2+(c+a)^2) - (b-c)(2a^2+(b+c)^2)}{(2a^2+(b+c)^2)(2b^2+(c+a)^2)} \\
    &= (a-b) \cdot \scriptstyle\frac{\left( a^3+a^2c+2ab^2-ac^2-2b^2c-c^3 \right) - \left( b^3+b^2c+2a^2b-bc^2-2a^2c-c^3 \right)}{(2a^2+(b+c)^2)(2b^2+(c+a)^2)} \\
    &= (a-b) \cdot \textstyle\frac{(a^3-b^3) + c(a^2-b^2) + 2ab(b-a) + c^2(b-a) + 2c(a^2-b^2)}{(2a^2+(b+c)^2)(2b^2+(c+a)^2)} \\
    &= (a-b)^2 \cdot \frac{a^2+ab+b^2 + c(a+b) -2ab - c^2 + 2c(a+b)}{(2a^2+(b+c)^2) (2b^2+(c+a)^2)} \\
    &= (a-b)^2 \cdot \frac{a^2-ab+b^2 + 3c(a+b)-c^2}{(2a^2+(b+c)^2) (2b^2+(c+a)^2)}
  \end{align*}
  Taking the cyclic sum of this quantity and dividing by two will yield
  the left-hand side of the inequality.

  At this point we definitely do not want to try and apply one of the criteria;
  even clearing denominators doesn't look too terrible,
  since we've already done a good chunk of the work.\footnote{And in general,
    if SOS it's not obvious how to proceed from here,
    you're still in a very good position to complete the expansion:
    you have only one extra denominator to deal with, and whatever you get will already be in SOS form.
    I suppose you can view this as a ``safety net''.}

  But we've actually chosen the addends here in a good way; assume $c = \min \{a,b,c\}$.
  Then we can drop the rightmost term $\frac{(c-a)(c-b)}{2c^2+(a+b)^2} \ge 0$,
  and this is what remains of the left-hand side.
  But when $c = \min \{a,b,c\}$ this is clearly nonnegative!
\end{soln}

\begin{proof}[Second Solution]
  We invoke the powerful identity we discovered earlier with Schur:
  \[ 2(a-b)(a-c) = (a-b)^2 + (a-c)^2 - (b-c)^2. \]

  Upon letting $X_a = \frac{1}{2a^2+(b+c)^2}$ the inequality rewrites as
  \[ \half \cycsum \left( X_b+X_c-X_a \right)(b-c)^2 \ge 0 \]

  If $a \ge b \ge c$ then $X_a + X_c - X_b$ is evidently nonnegative.
  (In fact, check that $X_c \ge X_b$.)
  But now $S_b$, $S_b+S_a$ and $S_b+S_c$ are all nonnegative, so once again we are done.
\end{proof}

\subsection{Much Sharper}
\begin{example}[Darij Grinberg]
  Show that for any positive reals $a,b,c$,
  \[ \frac{(b+c)^2}{a^2+bc} + \frac{(c+a)^2}{b^2+ca} + \frac{(a+b)^2}{c^2+ab} \ge 6. \]
\end{example}
\begin{soln}
  Begin again with algebra
  \begin{align*}
    \cycsum -2 + \frac{(b+c)^2}{a^2+bc} &= \cycsum \frac{b^2+c^2-2a^2}{a^2+bc} \\
    &= \cycsum \frac{b^2-a^2 + c^2-a^2}{a^2+bc} \\
    &= \cycsum \frac{b^2-a^2}{a^2+bc} + \frac{a^2-b^2}{b^2+ca} \\
    \intertext{At which point our lemma applies!}
    &= \cycsum (a^2-b^2) \left( \frac{1}{b^2+ca} - \frac{1}{a^2+bc} \right) \\
    &= \cycsum (a^2-b^2) \left( \frac{(a^2+bc)-(b^2+ca)}{(a^2+bc)(b^2+ca)} \right) \\
    &= \cycsum (a-b)^2(a+b) \cdot \frac{a+b-c}{(a^2+bc)(b^2+ca)}  \\
    &= \frac{1}{\cycprod (a^2+bc)} \cycsum (a-b)^2(a+b)(a+b-c)(c^2+ab)
  \end{align*}
  The equality case $(a,b,c) = (1,1,0)$ appears, so we must be precise.
  Assume that $a \ge b \ge c$, and further that $a \ge b+c$ (otherwise the result is obvious).
  By the fourth criterion\footnote{We are motivated to choose this criterion for two reasons.
    First, it is tight at the equality case; secondly,
    this weights the positive $S_b$ by $a^2$, which is ``large'' since $a \ge b+c$.}
  of \Cref{thm:sos} it suffices to prove that
  \begin{align*}
    b^2(b+c)(b+c-a)(a^2+bc) + a^2(a+c)(a+c-b)(b^2+ca) &\ge 0 \\
    \iff a^2(a+c)(a-b+c)(b^2+ca) &\ge b^2(b+c)(a-b-c)(a^2+bc).
  \end{align*}
  Again, keeping close in mind the equality case, we write $a-b+c \ge a-b-c$,
  reducing the inequality to $a^2(a+c)(b^2+ca) \ge b^2(b+c)(a^2+bc)$.
  But this follows writing the inequalities $a+c \ge b+c$
  and $a^2(b^2+ca) \ge b^2(a^2+bc)$ and multiplying them!
\end{soln}

\section{The Ravi Substitution}
When attacking a problem involving the sides of a triangle,
it is often beneficial to delay the Ravi substitution until after the inequality
has been written in SOS form.
Often by that point it is no longer necessary; especially with the following lemma:

\begin{lemma}
  \label{thm:trilemma}
  Let $a \ge b \ge c$ be the sides of a triangle. Then $S_a(b-c)^2 + S_b(c-a)^2 + S_c(a-b)^2$ is nonnegative whenever
  \[ S_a, S_b \ge 0 \quad \text{and} \quad b^2 S_b + c^2 S_c \ge 0. \]
\end{lemma}
\begin{proof}
  Drop the $S_a(b-c)^2$ term. It suffices to show that
  \[ S_b \cdot \left( \frac{a-c}{a-b} \right)^2 + S_c \ge 0 \]
  Note that $\frac{a-c}{a-b} = 1 + \frac{b-c}{a-b}$ is a decreasing function in $a$,
  so the inequality is sharpest when $a = b+c$ (since of course $a \le b+c$).
  This is precisely the condition in the problem.
\end{proof}

If worst comes to worst, we can still use the Ravi substitution
with the added comfort that the degree will be two less,
and the result will already be in SOS format.

\subsection{A Small Example}
\begin{example}
  Let $a,b,c$ be the sides of a triangle.
  Prove that \[ a^3(b-c) + b^3(c-a) + c^3(a-b) + 2(a^2b^2+b^2c^2+c^2a^2) \ge 2abc(a+b+c). \]
\end{example}

\begin{figure}[h]
  \centering
  \begin{asy}
  size(5cm);
  real[] M = {0,
    1, -1,
    2, -2, 2,
    -1, -2, -2, 1,
    0, 1, 2, -1, 0
  };
  dumbass(M, 4);
\end{asy}
\caption{The corresponding triangle}
\end{figure}

\begin{soln}
  Of course one can simply perform the Ravi substitution and simply expand, but that's too much work.

  After moving everything to one side, we can consider our given as the sum of the two triangles:

  \begin{figure}[h]
    \centering
    \begin{asy}
    size(9cm);
    real[] M1 = {0,
      0, -2,
      0, 2, 0,
      -2, 2, 2, 0,
      0, 0, 0, -2, 0
    };
    real[] M2 = {0,
      1, 1,
      2, -4, 2,
      1, -4, -4, 1,
      0, 1, 2, 1, 0
    };
    dumbass(M1, 4, (0, 0));
    dumbass(M2, 4, (12, 0));
    \end{asy}
    \caption{Forcing symmetry unto our inequality.}
  \end{figure}

  The first triangle is chosen so that the second is symmetric,
  and so that it can also be written as the sum:
  \[ -2 \cycsum ab(b-c)^2 \]
  The second triangle can then be simplified using standard techniques as
  \[ \cycsum a(b+c)(b-c)^2 + \cycsum a^2(b-c)^2 \]

  So the quantity in question is
  \[ \cycsum (ab+ac-2ab+a^2)(b-c)^2 = \cycsum a(c+a-b)(b-c)^2 \ge 0. \qedhere \]
\end{soln}

\subsection{Iran 1996}
Finally, to top off this article we have the infamous Iran TST 1996 inequality.
\begin{example}[Iran TST 1996]
  Let $x$, $y$, $z$ be positive reals. Prove that
  \[ \left( xy+yz+zx \right)\left( \frac{1}{(x+y)^2} + \frac{1}{(y+z)^2} + \frac{1}{(z+x)^2} \right) \ge \frac{9}{4}. \]
\end{example}
\begin{soln}
  Rather than expanding directly, we apply the Ravi Substitution \textit{in reverse};
  that is, $a=y+z$, $b=z+x$, $c=x+y$.
  This yields
  \[ \left( 2ab+2bc+2ca-a^2-b^2-c^2 \right)\left( \frac{1}{a^2} + \frac{1}{b^2} + \frac{1}{c^2} \right) \ge 9. \]
  This is much easier to expand! We get:
  \[ \cycsum b^2c^2(2ab+2bc+2ca-a^2-b^2-c^2) \ge 9a^2b^2c^2 \]
  Putting this in triangle form gives:
  \begin{figure}[h]
    \centering
    \begin{asy}
    size(8cm);
    real[] M = {
      0,
      0, 0,
      -1, 0, -1,
      2, 2, 2, 2,
      -1, 2, -12, 2, -1,
      0, 0, 2, 2, 0, 0,
      0, 0, -1, 2, -1, 0, 0
    };
    dumbass(M, 6);
    \end{asy}
    \caption{Iran, expanded}
  \end{figure}

  Taking out the $\{-1, 2, -1\}$ on the edges and then knocking out the hexagon,
  we see that we may write this as
  \[ -\cycsum b^2c^2(b-c)^2 + 2\cycsum a^2bc(b-c)^2 = \cycsum bc(2a^2-bc)(b-c)^2 \]
  Now applying \Cref{thm:trilemma}, it suffices to prove that
  \[ b^2 \cdot ca(2b^2-ca) + c^2 \cdot ab(2c^2-ab) \ge 0 \iff 2abc\left( b^3+c^3-abc \right) \ge 0 \]
  and this is trivial.
\end{soln}

\section{Summary}
Some subset of the main ideas:
\begin{itemize}
  \ii A polynomial inequality can be written in Chinese Dumbass notation,
    making it visually easy to see how sharp the inequality is.
  \ii If a polynomial inequality does not succumb to Schur and Muirhead,
    then SOS can yield a solution.
  \ii Cyclic (as opposed to symmetric) inequalities are no harder to put in SOS form.
  \ii With enough skill, SOS can be used without, before, or to simplify expansion.
  \ii SOS provides an alternative to the Ravi substitution in triangle-side inequalities.
\end{itemize}
Often I will expand an inequality with the hope of applying Schurhead,
and upon failing, fall back on SOS to provide stronger estimates.
But as we have seen, this is far from the only way to use SOS.

\section{Exercises}
\begin{enumerate}
  \ii Show that for nonnegative $a$, $b$, $c$,
    \[ 2(a^6+b^6+c^6) + 16(a^3b^3+b^3c^3+c^3a^3) \ge 9a^4(b^2+c^2) + 9b^4(c^2+a^2) + 9c^4(a^2+b^2). \]
  \ii Show that for all positive reals $x$, $y$, $z$,
    $\frac{x^3+xyz}{y+z} + \frac{y^3+xyz}{z+x} + \frac{z^3+xyz}{x+y} \ge x^2+y^2+z^2$.
  \ii (Wenyu Cao, ELMO 2009) Let $a$, $b$, $c$ be nonnegative real numbers.
    Prove that \[ a(a-b)(a-2b)+b(b-c)(b-2c)+c(c-a)(c-2a)\geq 0. \]
  \ii For positive reals $a$, $b$, $c$, prove that
    $\frac{a^{3}+b^{3}+c^{3}}{3abc}+\frac{a(b+c)}{b^{2}+c^{2}}
    +\frac{b(c+a)}{c^{2}+a^{2}}+\frac{c(a+b)}{a^{2}+b^{2}}\geq 4$.
  \ii (Michael Rozenberg) Show that for positive reals $a$, $b$, $c$,
    \[ \frac{a^2}{b+c} + \frac{b^2}{c+a} + \frac{c^2}{a+b} \ge \frac{3}{2} \cdot \frac{a^3+b^3+c^3}{a^2+b^2+c^2}. \]
  \ii (MOP 2003) Show that for $a,b,c \ge 0$ we have
  \[ \begin{array}{rccccccccccccc}& & & & & & 0\\\noalign{\smallskip}
    & & & & & 0 & & 0\\\noalign{\smallskip}
    & & & & 1 & &-2 & & 1\\\noalign{\smallskip}
    & & &-2 & & 2 & & 2 & &-2\\\noalign{\smallskip}
    & & 1 & & 2 & &-6 & & 2 & & 1\\\noalign{\smallskip}
    & 0 & &-2 & & 2 & & 2 & &-2 & & 0\\\noalign{\smallskip}
    0 & & 0 & & 1 & &-2 & & 1 & & 0 & & 0\\\noalign{\smallskip}
    \end{array} \ge 0. \]
\end{enumerate}

\begin{thebibliography}{99}
  \bibitem{sos} Pham Kim Hung, \textit{Sums of Squares}.
    \url{http://www.artofproblemsolving.com/Forum/viewtopic.php?f=55&t=80127}.
  \bibitem{cdn} Brian Hamrick, \textit{The Art of Dumbassing}.
    \url{http://www.tjhsst.edu/~2010bhamrick/files/dumbassing.pdf}.
  \bibitem{add} Thomas Mildorf, \textit{Inequalities - Addendum}.
    Handout at Red MOP 2011.
  \bibitem{mildorf} Thomas Mildorf, \textit{Olympiad Inequalities}.
    \url{http://www.artofproblemsolving.com/Resources/Papers/MildorfInequalities.pdf}.
\end{thebibliography}

\end{document}
