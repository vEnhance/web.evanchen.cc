\documentclass[11pt]{scrartcl}
\usepackage[sexy]{evan}
\usepackage{booktabs}
\usepackage{bbding}
%\providecommand{\isRp}{{\color{green!60!black}\CheckmarkBold\CheckmarkBold}}
\providecommand{\isRq}{{\color{green!60!black}\CheckmarkBold}}
\providecommand{\isUs}{{\color{cyan}\sffamily\bfseries O}}
\providecommand{\isAv}{{\color{gray}\sffamily\bfseries !!}}
\providecommand{\isEx}{{\color{red}\sffamily\bfseries X}}
\ohead{\bfseries\footnotesize Unofficial syllabus for math olympiads}

\begin{document}
\title{Unofficial syllabus for math olympiads}
\date{31 January 2023}
\maketitle

Inspired by the \href{https://ioinformatics.org/files/ioi-syllabus-2019.pdf}
{syllabus for the IOI},
the purpose of this document is to give some guidance
on what sorts of topics could appear on math olympiads.

\section{Disclaimer}
\alert{This document is not official guidance},
rather a first approximation based upon one person's experience.
There is no promise that the International Math Olympiad,
much less any particular country's exams,
will follow this syllabus.
Many top countries at the IMO routinely use problems
which may demand more knowledge.
In addition, the content of IMO exam changes over time as well
(albeit gradually) and in ways that are arbitrary and not easily predictable.

This list is \emph{not} intended to be an exhaustive enumeration;
it highlights guidelines for the ``most common cases''
and is deliberately ambiguous about certain topics.

\section{Introduction}
Each topic is classified into one of the following categories.

\begin{description}
  \ii[\isRq\ Required]
  Topics in this category are prerequisite knowledge for solutions,
  and all contestants are expected to know them.

  \ii[\isUs\ Useful for solution]
  Topics in this category are often useful in solutions
  and contestants are encouraged to be familiar with this topic.
  However, when a solution is related to this topic,
  often there is a different solution using only ``required'' knowledge,
  or else the solution would be reasonable even to non-experts.
  Terms from these topics will typically not appear
  in problem statements without definition.
  Therefore, these topics are less important than topics marked as ``required''.

  \ii[\isAv\ Advanced topic]
  Topics in this area may be helpful (or occasionally even necessary)
  in more difficult problems on the IMO.
  However, beginners should prioritize learning
  other topics before this one.
  Concepts from this area will typically not appear in
  problem statements without definition.

  \ii[\isEx\ Not required]
  Contestants are not expected to know this topic.
  Problems in which knowledge of these areas
  gives a substantial advantage are rare at the IMO.
  They may still be applicable in some situations; however in this case
  an alternate solution often exists that at least
  nominally does not use these concepts.
  Team selection tests from high-performing countries
  may not necessarily respect this classification.
\end{description}

%\section{General}

In general, not much knowledge is demanded of contestants.
Problem statements are largely meant to be understandable even to those
without much formal training.
Topics usually covered in advanced undergraduate studies are often excluded.
Rather, the problems are somewhat meant to
test ingenuity and intuition rather than mere knowledge,
and often demand simple tools to be used in unexpected ways.

All students are expected to be fluent with mathematical proofs.

\section{Topics in algebra}
\begin{center}
  \begin{tabular}{cp{12cm}}
  \toprule Status & Topic \\ \midrule
  \isRq & Elementary manipulation (e.g.\ factoring or expanding)
    of algebraic equations, expressions, inequalities \\
  \isRq & Familiarity with $\sum$ and $\prod$ notation \\
  \isRq & Definition of a function, and concepts such as injectivity and surjectivity \\
  \isRq & Cauchy's functional equation \\
  \isRq & Polynomials in one variable and basic properties
    like Vieta's relations or the fundamental theorem of algebra \\
  \isRq & Polynomials in two or more variables \\
  \isRq & Complex numbers, trigonometric functions, and their relations \\
  \isRq & AM-GM and Cauchy inequality \\
  \isAv & H\"{o}lder and Jensen inequality \\
  \isAv & Generating functions \\
  \isAv & Calculus in one or more variables \\
  % \isAv & Karamata and Muirhead inequality \\
  \isEx & Matrices, determinants, and other concepts from linear algebra \\
  \isEx & Real or complex analysis \\
  \isEx & Concepts from abstract algebra like groups, rings, fields \\
  \isEx & Algebraic geometry, e.g.\ nullstellensatz or B\'{e}zout theorem \\
  % \isAv & Lagrange multipliers \\
  \bottomrule
  \end{tabular}
\end{center}

\section{Topics in combinatorics}
\begin{center}
  \begin{tabular}{cp{12cm}}
  \toprule Status & Topic \\ \midrule
  \isRq & Basic counting arguments, e.g.\
    writing expressions like $2^n$, $n!$ or $\binom nk$ \\
  \isRq & Principle of mathematical induction \\
  \isRq & Recursion and recurrence relations \\
  \isRq & The pigeonhole principle \\
  \isRq & Definition of sets and functions \\
  \isUs & Elementary probability \\
  \isUs & Expected value and linearity of expectation \\
  \isUs & Basic properties and definitions from graph theory,
    e.g.\ connectedness and degree of a vertex \\
  %\isAv & Basic results from graph theory such as the existence of spanning trees
  %  or conditions for a bipartite graph \\
  %\isUs & First principles and definitions in game theory \\
  \isUs & Definition and existence of the convex hull of a finite set of points \\
  \isAv & Nontrivial results from graph theory,
    such as Hall's marriage lemma or Turan's theorem \\
  \isEx & Advanced enumerative identities such as the hook-length formula \\
  \isEx & Nontrivial probability theory, e.g.\ law of large numbers or martingales \\
  \isEx & Nontrivial theorems or machinery from game theory,
    such as Nash equilibriums, Sprague-Grundy, folk theorems, etc. \\
  \bottomrule
  \end{tabular}
\end{center}

\section{Topics in geometry}
\begin{center}
  \begin{tabular}{cp{12cm}}
  \toprule Status & Topic \\ \midrule
  \isRq & Definitions and basic properties
    of the incenter, centroid, orthocenter, circumcenter,
    excenters, and nine-point circle \\
  \isRq & Angle chasing and cyclic quadrilaterals \\
  \isRq & Similar triangles \\
  \isRq & Power of a point and radical axis / center \\
  \isRq & Homothety \\
  \isRq & Statement of Ceva and Menelaus theorems \\
  \isUs & Use of trigonometry, e.g.\ law of sines and cosines \\
  \isUs & Coordinate systems such as Cartesian coordinates,
    complex numbers, or barycentric coordinates \\
  \isUs & Inversive geometry \\
  \isUs & Projective geometry, e.g.\ cross ratios, harmonic bundles, poles and polars,
    Pascal's theorem, and so on \\
  \isUs & Spiral similarity \\
  \isAv & Definitions and basic properties of conic sections \\
  \isEx & Curves of degree higher than $2$ \\
  \isEx & Solid geometry \\
  \bottomrule
  \end{tabular}
\end{center}


\section{Topics in number theory}
\begin{center}
  \begin{tabular}{cp{12cm}}
  \toprule Status & Topic \\ \midrule
  \isRq & Basic results about primes like fundamental theorem of arithmetic \\
  \isRq & Modular arithmetic, and basic results like Fermat's little theorem,
    Chinese remainder theorem, modular inverses, orders modulo $n$ \\
  \isRq & Exponents in prime factorizations,
    and basic properties such as $\nu_p(xy) = \nu_p(x) + \nu_p(y)$
    or $\nu_p(x+y) \ge \min \left\{ \nu_p(x), \nu_p(y) \right\}$ \\
  \isRq & Euclidean algorithm and adjacent ideas,
    e.g.\ $\gcd(a,b) = \gcd(a-b,b)$ \\
  \isUs & Statement of Dirichlet's theorem on primes in arithmetic progression \\
  \isUs & Fermat's theorem on integers which are sums of two squares \\
  \isAv & Quadratic residues and quadratic reciprocity \\
  \isAv & Statement of the prime number theorem \\
  % \isAv & Factorization of polynomials over $\ZZ$ \\
  \isEx & Algebraic number theory, such as number fields, etc. \\
  \isEx & Analytic number theory, such as $L$-functions, etc. \\
  \bottomrule
  \end{tabular}
\end{center}

\section{Miscellaneous excluded topics}
\begin{center}
  \begin{tabular}{cp{12cm}}
  \toprule Status & Topic \\ \midrule
  \isEx & Mathematical modelling of real-life situations \\
  \isEx & Statistics \\
  \isEx & Theoretical computer science, e.g.\ complexity theory or NP-completeness \\
  \isEx & Foundational set theory and logic, e.g.\ ZFC axioms \\
  \isEx & Numerical approximation \\
  \isEx & Philosophy of mathematics \\
  \bottomrule
  \end{tabular}
\end{center}

\end{document}
