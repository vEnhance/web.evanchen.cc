\documentclass[11pt]{scrartcl}
\usepackage[sexy]{evan}

\begin{document}
\title{Lemmas in AoPS Geometry}
\subtitle{aka: how you can tell who's done too much olympiad geo}
\date{13 August 2023}
\maketitle

\begin{center}
  \includegraphics[width=0.95\textwidth]{slang.jpg}
\end{center}

\begin{abstract}
  This document is a compilation of a bunch of silly nicknames and terms
  that have become popular on the Internet among modern teenagers.
\end{abstract}

\section*{Important disclaimer (how to use this file)}
Before writing this handout, even I (Evan)
did not know several of the names of the points here.
If you are a beginner or intermediate level geometry student,
\alert{I do not recommend memorizing the names or configurations in this file}.
The reason is:
\begin{itemize}
  \ii They do not come up often enough to be worth learning for non-experts;
  \ii When they \emph{do} come up, experienced geometers can often figure
  the relevant facts on-the-spot anyway. Thus, the only real value is
  documenting the ``slang'' name for a fact that they could derive on their own.
\end{itemize}
Instead, this is meant to be an encyclopedic reference that you can refer to
when you hear some bizarre name on the Internet that does not appear
in the textbook EGMO.

\newpage

\tableofcontents

\newpage

\section{Basic configurations}
\subsection{Reim's theorem}
Reim's theorem is a nickname given for a certain easy angle chasing statement.
To fully state it, the following somewhat silly definition can be given.

\begin{definition}
  We say two lines $\ell_1$, $\ell_2$ are \emph{antiparallel}
  with respect to $m_1$ and $m_2$, if the four points $\ell_i \cap m_j$
  are the vertices of a cyclic quadrilateral.
\end{definition}

\begin{theorem}
  [Reim's theorem]
  Fix two lines $m_1$ and $m_2$.
  Prove that if $\ell_1$ is antiparallel to $\ell_2$
  and $\ell_2$ is antiparallel to $\ell_3$,
  then lines $\ell_1$ and $\ell_3$ are either
  parallel or coincide.
\end{theorem}
\begin{proof}
  Because $\dang(m_1, \ell_1) = -\dang(m_2, \ell_2) = \dang(m_1, \ell_3)$.
\end{proof}

See \Cref{fig:reim}; the angles are highlighted in green.
You'll notice this is literally a two-step angle chase,
which means that for practical purposes,
\alert{whenever you see the name ``Reim's theorem'',
it's just a shorthand for a certain two-step anglechase}.

So, if you ever come up to me with the excuse
``I didn't solve the problem because I didn't know Reim's theorem'',
I will be sassy in my reply.

\begin{figure}[ht]
  \centering
  \begin{asy}
    size(10cm);
    pair A = dir(110);
    pair B = dir(70);
    pair C = dir(30);
    pair D = dir(180);
    pair P = extension(A,D,B,C);
    real k = 5;
    pair U = (k+1)*A-k*P;
    pair V = (k+1)*B-k*P;

    draw(U--P--V, red+1.5);
    draw(A--B, blue+1.5);
    draw(C--D, blue+1.5);
    draw(U--V, blue+1.5);
    draw(unitcircle, grey);
    draw(circumcircle(U, V, C), grey);
    dot(P);
    dot(A);
    dot(B);
    dot(C);
    dot(D);
    dot(U);
    dot(V);

    path anglemark(pair A, pair B, pair C, real t=8) {
      pair M,N,P[],Q[];
      path mark;
      M=t*0.03*unit(A-B)+B;
      N=t*0.03*unit(C-B)+B;
      mark=arc(B,M,N);
      mark=(mark--B--cycle);
      return mark;
    }
    filldraw(anglemark(B,A,P,6), opacity(0.2)+yellow, deepgreen);
    filldraw(anglemark(P,C,D,6), opacity(0.2)+yellow, deepgreen);
    filldraw(anglemark(V,U,P,6), opacity(0.2)+yellow, deepgreen);
  \end{asy}
  \caption{The three angles of Reim's theorem.}
  \label{fig:reim}
\end{figure}

\subsection{The Incenter/Excenter Lemma / Fact 5 / Trillium theorem / chicken feet
theorem (EGMO Lemma 1.18)}

The following theorem from EGMO has several slang names.
The most popular one, ``Fact 5'', originated from the filename at
\url{https://web.evanchen.cc/handouts/Fact5/Fact5.pdf}
(despite the name itself not appearing in the handout at all).

\begin{lemma}
  [EGMO Lemma 1.18]
  Let $ABC$ be a triangle with incenter $I$, $A$-excenter $I_A$.
  The circumcenter of cyclic quadrilateral $IBI_AC$
  (that is, the midpoint of the diameter $\ol{II_A}$)
  coincides with the arc midpoint $L$ of minor arc $BC$.
\end{lemma}
\begin{figure}[ht]
  \centering
  \begin{asy}
    pair A = dir(110);
    pair B = dir(210);
    pair C = dir(330);
    pair I = incenter(A, B, C);
    draw(A--B--C--cycle);
    draw(unitcircle);
    pair L = dir(270);
    pair I_A = 2*L-I;
    filldraw(CP(L, I), opacity(0.2)+lightcyan, blue);

    draw(A--I_A, dotted);

    dot("$A$", A, dir(A));
    dot("$B$", B, dir(190));
    dot("$C$", C, dir(-10));
    dot("$I$", I, dir(60));
    dot("$L$", L, dir(225));
    dot("$I_A$", I_A, dir(I_A));

    /* Source generated by TSQ */
  \end{asy}
  \caption{EGMO Lemma 1.18, also known as ``Fact 5'',
  and many other regional nicknames.}
  \label{fig:fact5}
\end{figure}
\begin{proof}
  This is just angle chasing; see \Cref{fig:fact5}.
  Let $A = \angle BAC$, $B = \angle CBA$, $C = \angle ACB$,
  and note that $A$, $I$, $L$ are collinear (as $L$ is on the angle bisector).
  We are going to show that $LB = LI$, the other cases being similar.

  First, notice that
  \[ \angle LBI = \angle LBC + \angle CBI
    = \angle LAC + \angle CBI = \angle IAC + \angle CBI = \half A + \half B. \]
  However,
  \[ \angle BIL = \angle BAI + \angle ABI = \half A + \half B. \]
  Hence, $\triangle BIL$ is isosceles.
  So $LB = LI$.
  The rest of the proof proceeds along these lines.
\end{proof}
You can remember this lemma as saying that the
\alert{circumcenter of $BIC$ is the arc midpoint}.

\subsection{Iran lemma (EGMO Lemma 1.44)}
The following lemma in EGMO earned the nickname ``Iran lemma''
or ``Iran incenter lemma'' because of its prominence in Problem 11.19 of EGMO
(which was from an Iran TST, and from the last chapter of EGMO).
This lemma also featured as USAJMO 2014 problem 6.

\begin{figure}[ht]
  \centering
  \begin{asy}
    pair A = dir(130);
    pair B = dir(210);
    pair C = dir(330);
    pair I = incenter(A, B, C);
    pair D = foot(I, B, C);
    pair E = foot(I, C, A);
    pair F = foot(I, A, B);
    pair M = midpoint(B--C);
    pair N = midpoint(C--A);
    pair K = extension(M, N, E, F);

    filldraw(A--B--C--cycle, opacity(0.1)+lightcyan, blue);
    draw(incircle(A, B, C), blue);
    draw(M--K, red);
    draw(F--K, red);
    draw(B--K--C, red);

    dot("$A$", A, dir(A));
    dot("$B$", B, dir(B));
    dot("$C$", C, dir(C));
    dot("$I$", I, dir(315));
    dot("$D$", D, dir(D));
    dot("$E$", E, dir(70));
    dot("$F$", F, dir(F));
    dot("$M$", M, dir(M));
    dot("$N$", N, dir(N));
    dot("$K$", K, dir(K));

    /* -----------------------------------------------------------------+
    |                 TSQX: by CJ Quines and Evan Chen                  |
    | https://github.com/vEnhance/dotfiles/blob/main/py-scripts/tsqx.py |
    +-------------------------------------------------------------------+
    A = dir 130
    B = dir 210
    C = dir 330
    I 315 = incenter A B C
    D = foot I B C
    E 70 = foot I C A
    F = foot I A B
    M = midpoint B--C
    N = midpoint C--A
    K = extension M N E F
    A--B--C--cycle / 0.1 lightcyan / blue
    incircle A B C / blue
    M--K / red
    F--K / red
    B--K--C / red
    */
  \end{asy}
  \caption{The Iran incenter lemma.}
  \label{fig:iran}
\end{figure}


\begin{lemma}
  [EGMO Lemma 1.44]
  The incircle of $\triangle ABC$ is tangent to $\ol{BC}$, $\ol{CA}$, $\ol{AB}$
  at $D$, $E$, $F$, respectively.
  Let $M$ and $N$ be the midpoints of $\ol{BC}$ and $\ol{AC}$, respectively.
  Ray $BI$ meets line $EF$ at $K$.
  Then $\ol{BK} \perp \ol{CK}$ and $K$ lies on $MN$.
\end{lemma}
\begin{proof}
  See \Cref{fig:iran}.
  Left as exercise because it's a problem in EGMO.
  Hint: focus on the circle with diameter $\ol{IC}$,
  which passes through $D$, $E$, and $K$.
\end{proof}

A good way to remember this lemma is that:
\begin{quote}
\alert{The $A$-touch chord, $B$-bisector, and $C$-midline concur}.
\end{quote}


\subsection{Shooting lemma / Death Star lemma (EGMO Lemma 4.33)}
Lemma 4.33 of EGMO concerns the picture in which you have two internally
tangent circles and a segment of the large circle tangent to the smaller one.
In Brazil, it is called the \alert{Death Star lemma} because
the two circles look like the Death Star spaceship from Star Wars.
Here it is:
\begin{lemma}[EGMO Lemma 4.33]
  Let $\ol{AB}$ be a chord of a circle $\Omega$.
  Let $\omega$ be a circle tangent to chord $\ol{AB}$ at $K$
  and internally tangent to $\omega$ at $T$.
  Then ray $TK$ passes through the midpoint $M$ of the arc $\widehat{AB}$ not containing $T$.
  Moreover, \[ MA^2 = MB^2 = MK \cdot MT. \]
\end{lemma}

\begin{figure}[ht]
  \centering
  \begin{asy}
    size(6cm);
    pair A = dir(170);
    pair B = dir(10);
    pair M = dir(270);
    pair T = dir(55);
    pair K = extension(M, T, A, B);
    filldraw(unitcircle, opacity(0.1)+cyan, blue);
    pair V = K+dir(90);
    pair P = extension(origin, T, K, V);
    filldraw(CP(P, K), opacity(0.1)+yellow, deepgreen);
    draw(A--B, blue);
    draw(M--T, red);
    draw(B--M--A, grey);

    dot("$A$", A, dir(A));
    dot("$B$", B, dir(B));
    dot("$M$", M, dir(M));
    dot("$T$", T, dir(T));
    dot("$K$", K, dir(315));
    dot("$P$", P, dir(P));

    /* -----------------------------------------------------------------+
    |                 TSQX: by CJ Quines and Evan Chen                  |
    | https://github.com/vEnhance/dotfiles/blob/main/py-scripts/tsqx.py |
    +-------------------------------------------------------------------+
    A = dir 170
    B = dir 10
    M = dir 270
    T = dir 55
    K 315 = extension M T A B
    unitcircle / 0.1 cyan / blue
    V := K+dir(90)
    P = extension origin T K V
    CP P K / 0.1 yellow / deepgreen
    A--B / blue
    M--T / red
    B--M--A / grey
    */
  \end{asy}
  \caption{The shooting lemma / death star lemma.}
  \label{fig:shooting}
\end{figure}

\begin{proof}
  See \Cref{fig:shooting}.
  The fact that $TK$ passes through the arc midpoint
  follows by taking the homothety at $T$ mapping $\omega$ to $\Omega$:
  it should map the ``south pole'' of $\omega$
  to that of $\Omega$, ergo it maps $M$ to $K$.

  Meanwhile, $MK \cdot MT = MA^2$ follows from
  $\triangle MKA \sim \triangle MAT$.
\end{proof}
In the United States, the name \alert{shooting lemma} refers to various
parts of this lemma (I never figured out which).

\subsection{Salmon theorem}
The name ``Salmon theorem'' refers to a certain easy spiral similarity.
It appears to have first been used in the official solutions for USAMO 2013
provided by the MAA. Here Salmon is a last name, not a kind of fish.

\begin{figure}[ht]
  \centering
  \begin{asy}
    pair A = dir(125);
    pair B = dir(210);
    pair C = dir(330);
    pair P = 0.3*C+0.7*B;
    filldraw(A--B--C--cycle, opacity(0.1)+lightcyan, blue);

    pair O_1 = circumcenter(A, B, P);
    pair O_2 = circumcenter(A, C, P);

    real r1 = abs(P-O_1);
    real r2 = abs(P-O_2);
    pair Se = (r1*O_2-r2*O_1)/(r1-r2);
    path w1 = CP(O_1,P);
    path w2 = CP(O_2,P);
    filldraw(w1, opacity(0.1)+yellow, deepgreen);
    filldraw(w2, opacity(0.1)+yellow, deepgreen);
    filldraw(A--O_1--O_2--cycle, opacity(0.1)+lightred, red);
    draw(A--P, grey);

    dot("$A$", A, dir(A));
    dot("$B$", B, dir(B));
    dot("$C$", C, dir(C));
    dot("$P$", P, dir(-90));
    dot("$O_1$", O_1, dir(-90));
    dot("$O_2$", O_2, dir(-90));
  \end{asy}
  \caption{Salmon's theorem.}
  \label{fig:salmon}
\end{figure}

\begin{theorem}
  [Salmon theorem]
  Let $ABC$ be a triangle and $P$ a point on line $BC$.
  Then we have the spiral similarity
  \[ \triangle AO_1B \overset{+}{\sim} \triangle AO_2C, \qquad
    \triangle ABC \overset{+}{\sim} \triangle AO_1O_2. \]
\end{theorem}
\begin{proof}
  See \Cref{fig:salmon}.
  Assume without loss of generality that $\angle APB \le 90\dg$.
  Then \[ \angle AO_1B = 2 \angle APB \]
  but \[ \angle AO_2C = 2 \left( 180 - \angle APC \right) = 2 \angle ABP. \]
  Hence $\angle AO_1B = \angle AO_2C$.
  Moreover, both triangles are isosceles, establishing the first similarity.

  The second part follows from spiral similarities coming in pairs.
\end{proof}

Confusingly, there are other results named ``Salmon theorem''.
See, e.g.,
\href{https://www.awesomemath.org/wp-pdf-files/math-reflections/mr-2013-06/hartcourt_theorem.pdf}{this result}.

\subsection{The first isogonality lemma}
The following lemma appears in \emph{Geometry Revisited},
Section 1.9, Exercise 3.
It comes up once in a blue moon, and has earned the name ``isogonality lemma''.
\begin{lemma}
  [The first isogonality lemma]
  Let $ABC$ be a triangle and let $P$ be a point inside
  it satisfying $\angle ABP = \angle PCA$.
  Let $Q$ be the reflection of $P$ across the midpoint of $\overline{BC}$.
  Then $\angle BAP = \angle CAQ$.
\end{lemma}
\begin{proof}
  Most straightforward would be to use barycentric coordinates,
  but you can see a synthetic proof at \url{https://aops.com/community/c6h518987}.
\end{proof}

The condition $\angle ABP = \angle ACP$ is actually
usually more naturally interpreted as saying the points
$B$, $C$, $\ol{BP} \cap \ol{AC}$ and $\ol{CP} \cap \ol{AB}$ are cyclic.
Similarly, the parallelogram $BPCQ$ is noteworthy.

\begin{figure}[ht]
  \centering
  \begin{asy}
  size(7cm);
  pair Y = dir(70);
  pair Z = dir(130);

  pair B = dir(170);
  pair C = dir(10);

  pair A = extension(B, Z, C, Y);
  pair P = extension(B, Y, C, Z);
  pair Q = B+C-P;
  filldraw(A--B--C--cycle, opacity(0.1)+lightgreen, deepgreen);
  filldraw(B--P--C--Q--cycle, opacity(0.1)+yellow, heavygreen);
  draw(P--Q, heavygreen);
  draw(Z--P--Y, blue+dashed);

  filldraw(circumcircle(B, Z, Y), opacity(0.1)+lightcyan, deepcyan+dashed);

  draw(P--A--Q, blue);

  dot(Y);
  dot(Z);
  dot("$B$", B, dir(B));
  dot("$C$", C, dir(C));
  dot("$A$", A, dir(A));
  dot("$P$", P, dir(250));
  dot("$Q$", Q, dir(Q));

  clip(box((-1.2,-0.4),(1.2,2)));

  /* TSQ Source:

  Y .= dir 70
  Z .= dir 130

  B = dir 170
  C = dir 10

  A = extension B Z C Y
  P = extension B Y C Z R250
  Q = B+C-P
  A--B--C--cycle 0.1 lightgreen / deepgreen
  B--P--C--Q--cycle 0.1 yellow / heavygreen
  P--Q heavygreen
  Z--P--Y blue dashed

  circumcircle B Z Y 0.1 lightcyan / deepcyan dashed

  P--A--Q blue

  */
  \end{asy}
  \label{fig:isog}
  \caption{The first isogonality lemma.}
\end{figure}

\begin{remark}
  The \emph{second} isogonality lemma refers to the following statement:
  Suppose that $\ol{AP}$ and $\ol{AQ}$
  are isogonal with respect to $\angle A$.
  Let $X = \ol{PB} \cap \ol{QC}$ and $Y = \ol{PC} \cap \ol{QB}$.
  Then $\ol{AX}$ and $\ol{AY}$ are isogonal with respect to $\angle A$.
\end{remark}

\section{Length-based lemmas}
\subsection{Ratio lemma}
In the United States, the following fact is sometimes called the ratio lemma.
Like Reim's theorem, this name is ``unnecessary'' in that it is only
capturing a single natural step.

\begin{figure}[ht]
  \centering
  \begin{asy}
    size(6cm);
    pair A = dir(125);
    pair B = dir(210);
    pair C = dir(330);
    pair D = 0.3*C+0.7*B;
    filldraw(A--B--C--cycle, opacity(0.1)+lightcyan, blue);
    draw(A--D, grey);
    dot("$A$", A, dir(A));
    dot("$B$", B, dir(B));
    dot("$C$", C, dir(C));
    dot("$D$", D, dir(-90));
    markangle(1,19.0,B,A,D,deepgreen);
    markangle(2,12.0,D,A,C,deepgreen);
  \end{asy}
  \caption{The simple picture for the Ratio Lemma.}
  \label{fig:ratio}
\end{figure}

\begin{lemma}
  [Ratio lemma]
  Let $ABC$ be a triangle and $D$ a point in the interior of segment $BC$. Then
  \[ \frac{BD}{DC} = \frac{AB}{AC} \frac{\sin \angle BAD}{\sin \angle CAD}. \]
\end{lemma}
\begin{proof}
  See \Cref{fig:ratio}.
  By the law of sines on $\triangle ABD$ and $\triangle ACD$, we have
  \[
    \frac{BD}{\sin \angle BAD} = \frac{AB}{\sin \angle ADB}, \qquad
    \frac{CD}{\sin \angle CAD} = \frac{AC}{\sin \angle ADC}.
  \]
  Since $\angle ADB + \angle ADC = 180\dg$, cancelling those sines finishes.
\end{proof}
The lemma is most prominently featured in the handout posted by \texttt{mira74}
at \url{https://aops.com/community/p19166714}

\subsection{Prism lemma}
The following simple lemma from projective geometry is useful for
transferring cross ratios.
The name seems to have originated from an OTIS handout,
because the picture looks like a pyramid (see \Cref{fig:prism}).
\begin{figure}[ht]
  \centering
  \begin{asy}
    pair D = (0,4);
    pair A1 = (-1,0);
    pair A2 = (2,0);
    pair C1 = 0.7*D+0.3*A1;
    pair B1 = 0.5385*D+0.4615*A1;
    pair T = (6,0);
    pair B2 = extension(T, B1, D, A2);
    pair C2 = extension(T, C1, D, A2);
    draw(A1--D--A2, blue);
    draw(T--A1, lightred);
    draw(T--B1, lightred);
    draw(T--C1, lightred);
    dot(A1);
    dot(A2);
    dot(B1);
    dot(B2);
    dot(C1);
    dot(C2);
    dot(D);
    dot(T);
  \end{asy}
  \caption{The prism lemma.}
  \label{fig:prism}
\end{figure}

\begin{lemma}
  [Prism lemma]
  Two lines $\ell_1$ and $\ell_2$ meet at a point $D$.
  Let $A_1$, $B_1$, $C_1$ be distinct points on $\ell_1$
  and $A_2$, $B_2$, $C_2$ be distinct points on $\ell_2$, different from $D$.
  Then $A_1 A_2$, $B_1 B_2$, $C_1 C_2$ are concurrent
  if and only if $(A_1B_1; C_1D) = (A_2B_2; C_2D)$.
\end{lemma}
\begin{proof}
  This is a restatement of the fact that cross ratios are preserved
  under projection through a point.
\end{proof}

\subsection{EFFT (EGMO Theorem 7.16 and 7.25)}
In barycentric coordinates, there are two theorems in EGMO
that let one determine whether two displacement vectors are perpendicular.
Teenager Evan, in 2012, thought it was a good idea to call these
``EFFT'' and ``strong EFFT'' respectively,
in his handout that was posted on Art of Problem Solving.
Subsequent efforts to get rid of those names from the slang repository
have been largely unsuccessful, and Evan has given up.

\begin{theorem}[EGMO Lemma 7.16]
  Let $\overrightarrow{MN} = (x_1, y_1, z_1)$
  and $\overrightarrow{PQ} = (x_2, y_2, z_2)$ be displacement vectors.
  Then $\ol{MN} \perp \ol{PQ}$ if and only if
  \[  0 = a^2(z_1y_2 + y_1z_2) + b^2(x_1z_2 + z_1x_2) + c^2(y_1x_2 + x_1y_2). \]
\end{theorem}

\begin{theorem}[EGMO Lemma 7.25]
   Suppose $M$, $N$, $P$, and $Q$ are points with
   \begin{align*}
      \overrightarrow{MN} &= x_1 \overrightarrow{AO} + y_1 \overrightarrow{BO} + z_1 \overrightarrow{CO} \\
      \overrightarrow{PQ} &= x_2 \overrightarrow{AO} + y_2 \overrightarrow{BO} + z_2 \overrightarrow{CO}
   \end{align*}
   such that either $x_1+y_1+z_1=0$ or $x_2+y_2+z_2=0$.

   In that case, lines $MN$ and $PQ$ are perpendicular if and only if
   \[  0 = a^2(z_1y_2 + y_1z_2) + b^2(x_1z_2 + z_1x_2) + c^2(y_1x_2 + x_1y_2). \]
\end{theorem}


\section{Special named points inside a triangle}
For this section, we only provide the definition of the points
and leave the (many) properties of the handouts to provided external links.

\subsection{HM/Humpty and Dumpty point}
As shown in \Cref{fig:HM}, let $ABC$ be a triangle with orthic triangle $DEF$,
and orthocenter $H$. Let $M$ be the midpoint of $\ol{BC}$.
\begin{definition}
  The foot of the altitude from $H$ to $\ol{AM}$,
  denoted $Q$ in \Cref{fig:HM},
  is called either the \alert{$A$-Humpty point} or the \alert{$A$-HM point}.
\end{definition}
Some basic properties of the $A$-Humpty point are that lines $QH$, $BC$, $EF$
are concurrent, and that it lies on the circle $(BHC)$.
\begin{figure}[ht]
  \centering
  \begin{asy}
    size(10cm);
    pair A = dir(110);
    pair B = dir(210);
    pair C = dir(330);
    pair M = midpoint(B--C);
    pair H = orthocenter(A, B, C);
    pair G = foot(A, M, H);
    pair Q = foot(H, A, M);
    pair D = foot(A, B, C);
    pair E = foot(B, C, A);
    pair F = foot(C, A, B);

    filldraw(A--B--C--cycle, opacity(0.1)+lightred, red);
    filldraw(unitcircle, opacity(0.1)+yellow, red);
    draw(A--D, red);
    draw(B--E, red);
    draw(C--F, red);

    pair T = extension(E, F, B, C);
    draw(A--M, red);
    draw(A--T--Q, orange+dashed);
    draw(E--T--B, red);
    draw(G--M, orange+dashed);

    filldraw(circumcircle(A, E, F), opacity(0.1)+yellow, orange);

    dot("$A$", A, dir(A));
    dot("$B$", B, dir(B));
    dot("$C$", C, dir(C));
    dot("$M$", M, dir(M));
    dot("$H$", H, dir(H));
    dot("$Q$", Q, dir(335));
    dot("$D$", D, dir(D));
    dot("$E$", E, dir(20));
    dot("$F$", F, dir(F));
    // dot("$T$", T, dir(T));
    dot(T);
    dot("$G$", G, dir(G));

    /* TSQ Source:

    A = dir 110
    B = dir 210
    C = dir 330
    M = midpoint B--C
    H = orthocenter A B C
    G = foot A M H
    Q = foot H A M R335
    D = foot A B C
    E = foot B C A R20
    F = foot C A B

    A--B--C--cycle 0.1 lightred / red
    unitcircle 0.1 yellow / red
    A--D red
    B--E red
    C--F red

    T = extension E F B C
    A--M red
    A--T--Q orange dashed
    E--T--B red
    G--M orange dashed

    circumcircle A E F 0.1 yellow / orange

    */
  \end{asy}
  \caption{The Humpty point (denoted $Q$) and the $A$-queue point (denoted $G$).}
  \label{fig:HM}
\end{figure}

Less commonly seen is:
\begin{definition}
  The \alert{$A$-Dumpty point} is the isogonal conjugate of the $A$-Humpty point.
  It turns out to coincide with the midpoint of the $A$-symmedian chord.
\end{definition}

References for these points include:
\begin{itemize}
  \ii \url{https://math.stackexchange.com/a/3894697/229197}
  \ii \href{https://pregatirematematicaolimpiadejuniori.files.wordpress.com/2018/05/two_special_points__1_.pdf}
  {On Two Special Points in a Triangle},
  which seems to be the origin of the name Humpty and Dumpty.
  \ii \href{https://www.awesomemath.org/wp-pdf-files/math-reflections/mr-2017-02/article_1_a_special_point_on_the_median.pdf}
  {A Special Point on the Median}
  which seems to be the origin of the name HM.
\end{itemize}

\subsection{Queue points}
In the Humpty configuration, the point $G$ in \Cref{fig:HM}
also plays an important role.
It is named as follows:
\begin{definition}
  The \alert{$A$-queue point} is the second intersection of $(AEF)$ and $(ABC)$,
  denoted $G$ in \Cref{fig:HM}.
  In other words, it is the Miquel point of the cyclic quadrilateral $BFEC$.
\end{definition}
The most important properties of the $A$-queue point are:
\begin{theorem}
  [Important properties of the $A$-queue point]
  In the notation of \Cref{fig:HM}:
  \begin{enumerate}
    \ii Line $AG$ is concurrent with lines $EF$, $HQ$, $BC$ defined before.
    \ii Point $G$ lies on line $HM$.
  \end{enumerate}
\end{theorem}
The name seems to have come from the blog post by \texttt{math\_pi\_rate} at
\url{https://aops.com/community/c701535h1754217}\footnote{The
  concurrence point $\ol{AG} \cap \ol{EF} \cap \ol{HQ} \cap \ol{BC}$
  is named the $A$-ex point in that blog post,
  but that name seems to not have caught on.},
which documents several additional properties of the line.

\subsection{Sharky-Devil point}
As shown in \Cref{fig:sharkydevil},
let $ABC$ be a triangle with intouch triangle $DEF$.
\begin{definition}
  The \alert{$A$-Sharkydevil point}, denoted $K$ below,
  is defined as the Miquel point $K$ of $BFEC$;
  i.e.\ the intersection of the $(AEF)$ and $(ABC)$.
\end{definition}

\begin{figure}[ht]
  \centering
  \begin{asy}
  size(9cm);
  pair A = dir(110);
  pair B = dir(210);
  pair C = dir(330);
  filldraw(unitcircle, opacity(0.1)+palecyan, lightblue);
  draw(A--B--C--cycle, lightblue);
  pair I = incenter(A, B, C);
  pair D = foot(I, B, C);
  draw(incircle(A, B, C), lightblue);
  pair M = dir(270);
  pair K = extension(-A, I, M, D);
  pair E = foot(I, A, C);
  pair F = foot(I, A, B);
  draw(circumcircle(A, E, F), heavygreen);
  pair S = -A;
  pair P = foot(D, E, F);
  draw(I--K, lightblue);
  draw(K--M, heavygreen);
  draw(B--K--C, lightblue);
  draw(D--E--F--cycle, heavycyan);
  draw(D--P, heavycyan);

  dot("$A$", A, dir(A));
  dot("$B$", B, dir(B));
  dot("$C$", C, dir(C));
  dot("$I$", I, dir(I));
  dot("$D$", D, dir(D));
  dot("$M$", M, dir(M));
  dot("$K$", K, dir(K));
  dot("$E$", E, dir(E));
  dot("$F$", F, dir(F));
  // dot("$S$", S, dir(S));
  dot("$P$", P, dir(P));

  /* TSQ Source:

  A = dir 110
  B = dir 210
  C = dir 330
  unitcircle 0.1 palecyan / lightblue
  A--B--C--cycle lightblue
  I = incenter A B C
  D = foot I B C
  incircle A B C lightblue
  M = dir 270
  K = extension -A I M D
  E = foot I A C
  F = foot I A B
  circumcircle A E F heavygreen
  S = -A
  P = foot D E F
  S--K lightblue
  K--M heavygreen
  B--K--C lightblue
  D--E--F--cycle heavycyan
  D--P heavycyan

  */
  \end{asy}
  \caption{The $A$-Sharkydevil point.}
  \label{fig:sharkydevil}
\end{figure}

The name seems coined by \texttt{aops29} in a blog post,
\url{https://aops.com/community/c946900h1911664}.

The most important properties are:
\begin{theorem}
  [Important properties of the Sharkydevil point]
  In the notation of \Cref{fig:sharkydevil}:
  \begin{enumerate}
    \ii Line $\ol{KD}$ bisects $\angle BKC$
    and hence passes through the midpoint $M$ of $\arc{BC}$.
    \ii Line $\ol{KD}$ passes through
    the exsimilicenter of the incircle and circumcircle.
    \ii Let $P$ denote the foot from $D$ to $\ol{EF}$.
    Then $K$ is the inverse of $P$ with respect to the incircle.
  \end{enumerate}
\end{theorem}

\section{’Muricaaaaaaa, by \texttt{i3435}}
I tried to keep this handout brief and to-the-point,
since it is intended to be a lookup of common idiosyncratic names.

If you want a more complete reference,
there is a 2021 handout by \texttt{i3435} at
\begin{center}
  \url{https://aops.com/community/p20459825}
\end{center}
which is \emph{much} more comprehensive, including proofs of the facts here,
several properties of the named points not recorded here,
and many more configurations.
It also features a long set of problems and solutions.

\end{document}
