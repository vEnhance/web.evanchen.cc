\documentclass[11pt]{scrartcl}

%%%%%%%%%%%%%% LATEX SAMPLE FILE %%%%%%%%%%%%%%%%
% A line which starts with a % sign
% is called a COMMENT. It is IGNORED
% by the LaTeX processor.

% Include math
\usepackage{amsmath,amsthm,amssymb}
% Include links
\usepackage{hyperref}


%%%%%%%%%%%%%  THEOREMS  %%%%%%%%%%%%%%%%%
% Let's define some theorem environments
% To use later in the paper
\theoremstyle{plain} % other options: definition, remark
\newtheorem{theorem}{Theorem}
\newtheorem{lemma}[theorem]{Lemma}
% By including [theorem], the lemma follows the numbering of theorem
% e.g. Thm 1, Lemma 2, Thm 3, Thm 4, \dots
\theoremstyle{definition}
\newtheorem*{definition}{Definition} % the star prevents numbering

% Remarks
\theoremstyle{remark}
\newtheorem{remark}{Remark}


%%%%%%%%%%%%%%  PAGE SETUP %%%%%%%%%%%%%%%%%
% LaTeX has big default margins
% The following sets them to 1in
\usepackage[margin=1.5in]{geometry}

% The following sets up some headers
\usepackage{fancyhdr}
\pagestyle{fancy}
\lhead{An Example LaTeX Document} % Left Header
\rhead{\thepage} % Right Header
\cfoot{} % Center Foot (empty)

%%%%%%%%%%%%% SHORTCUTS %%%%%%%%%%%%%%%%%%%%
% You can define your own shortcuts too.
% Examples of custom commands
\newcommand{\half}{\frac{1}{2}}
\newcommand{\cbrt}[1]{\sqrt[3]{#1}}

% Document content begins here
\begin{document}

% Set up a title
\title{An Example LaTeX Document}
\author{Evan Chen}
\date{\today}
\maketitle

\begin{abstract}
  This is an example of a \LaTeX\ document,
  complete with theorems and headers and the like.
\end{abstract}

% This line makes a ToC
\tableofcontents

% This line starts a new page
\eject

\begin{quote}
  \textbf{Golden Rule}: Do not do anything manually if you can avoid it.
\end{quote}

I have seen people manually add section numbers, manually bold theorems,
manually create numbered lists, manually add spacing
between every single paragraph\dots
Do not do this. If you do not know how to do something, Google it!
The site \href{http://tex.stackexchange.com}{TeX.SE} will frequently provide answers.
Often, \LaTeX\ can do it, and can do it better than you can.

The philosophy here is that what you see is what you mean.
The source code shows you the logical structure of the document.
Then the compiler converts that to a PDF.

\section{Paragraphs}
In \LaTeX, paragraphs are caused when two line breaks are used.
Single line breaks are ignored.
Hence this entire block is one paragraph.
This is useful because it means you can break text at convenient points,
which makes your source code much more readable than
if you had lines spanning hundreds of characters.

Now this is a new paragraph. If you want to
start a new line without a new paragraph, use
two backslashes like this:
\\
Now the next words will be on a new line.
\textbf{As a general rule, use this as infrequently as possible.}

You can \textbf{bold} or \textit{italicize} text.
Try to not do so repeatedly for mechanical tasks by,
e.g.\ using theorem environments (see Section \ref{sec:theorem}).


\section{Math}
Inline math is created with dollar signs,
like $e^{i \pi} = -1$ or $\half \cdot 2 = 1$.

Display math is created as follows:
\[ \sum_{k=1}^n k^3 = \left( \sum_{k=1}^n k \right)^2. \]
This puts the math on a new line.
Remember to properly add punctuation to the
end of your sentences -- display math is
considered part of the sentence too!

Note that the use of \verb+\left(+
causes the parentheses to be the correct size.
Without them, get something ugly like
\[ \sum_{k=1}^n k^3 = ( \sum_{k=1}^n k )^2. \]

\subsection{Using alignment}
Try this:
\begin{align*}
  \prod_{k=1}^4 \left( i-x_k \right)\left( i+x_k \right) &= P(i) \cdot P(-i) \\
  &= \left( 1-b+d+i(c-a) \right)\left( 1-b+d-i(c-a) \right) \\
  &= (a-c)^2 + \left( b-d-1 \right)^2.
\end{align*}

\section{Shortcuts}
In the beginning of the document we wrote
\begin{verbatim}
\newcommand{\half}{\frac{1}{2}}
\newcommand{\cbrt}[1]{\sqrt[3]{#1}}
\end{verbatim}
Now we can use these shortcuts.
\[ \half + \half = 1 \text{ and } \cbrt{8} = 2. \]

\section{Theorems and Proofs}
\label{sec:theorem}
% ^ Now we can refer to this
Let us use the theorem environments we had in the beginning.
\begin{definition}
  Let $\mathbb R$ denote the set of real numbers.
\end{definition}
Notice how this makes the source code much more readable.

\begin{theorem}
  [Vasc's Inequality]
  \label{thm:vasc}
  For any $a$, $b$, $c$ we have the inequality
  \[ \left( a^2+b^2+c^2 \right)^2 \ge 3\left( a^3b+b^3c+c^3a \right). \]
\end{theorem}

For the proof of Theorem \ref{thm:vasc}, we need the following lemma.

\begin{lemma}
  We have $\left( x+y+z \right)^2 \ge 3(xy+yz+zx)$ for any $x,y,z \in \mathbb R$.
\end{lemma}
\begin{proof}
  This can be rewritten as
  \[ \half\left( (x-y)^2+(y-z)^2+(z-x)^2 \right) \ge 0 \]
  which is obvious.
\end{proof}

\begin{proof}
  [Proof of Theorem \ref{thm:vasc}]
  In the lemma, put $x=a^2-ab+bc$, $y=b^2-bc+ca$, $z=c^2-ca+ab$.
\end{proof}

\begin{remark}
  In \autoref{thm:vasc}, equality holds if
  $a : b : c = \cos^2 \frac{2\pi}{7} : \cos^2 \frac{4\pi}{7} : \cos^2 \frac{6\pi}{7}$.
  This unusual equality case makes the theorem difficult to prove.
\end{remark}

\section{Referencing}
The above examples are the simplest cases.
You can get much fancier: check out
\href{http://en.wikibooks.org/wiki/LaTeX/Labels_and_Cross-referencing}{the Wikibooks}.

\section{Numbered and Bulleted Lists}
Here is a numbered list.
\begin{enumerate}
  \item The environment name is ``enumerate''.
  \item You can nest enumerates.
    \begin{enumerate}
      \item Subitem
      \item Another subitem
    \end{enumerate}
  \item[$2 \half$.] You can also customize any particular label.
  \item But the labels continue onwards afterwards.
\end{enumerate}

\bigskip

You can also create a bulleted list.
\begin{itemize}
  \item The syntax is the same as ``enumerate''.
  \item However, we use ``itemize'' instead.
\end{itemize}

\section{More Sources}
This is just a modest example of some common usages. For more sources,
\begin{enumerate}
  \item The excellent Wikibooks,
    \url{http://en.wikibooks.org/wiki/LaTeX},
    provide a much more thorough, comprehensive treatise on using \LaTeX.
  \item Detexify, \url{http://detexify.kirelabs.org/classify.html},
    can be useful for locating a particular symbol.
  \item And finally, Google. Seriously.
    If you want to know how to do something,
    a quick Google search suffices 90\% of the time.
\end{enumerate}

\end{document}
